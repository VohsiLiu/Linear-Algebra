\section{线性空间与线性变换}

\subsection{线性空间的定义}
    \begin{definition}[数环]
        设$\mathfrak{R}$是非空数集,其中任何两个数之和、差与积仍属于$\mathfrak{R}$(即$\mathfrak{R}$关于加、减、乘法运算是封闭的),则称$\mathfrak{R}$是一个{\heiti 数环}。
    \end{definition}

    \begin{definition}[数域]
        若$K$是至少含有两个互异数的数环,且其中任何两数$a$与$b$之商$(b\neq 0)$仍属于$K$,则称$K$是一个{\heiti 数域}。
    \end{definition}

    \begin{definition}[线性空间]
        设$V$是一个非空集合,$K$是一个数域,$V$满足以下两个条件:
        \begin{enumerate}[(1)]
            \item 在$V$中定义一个封闭的加法运算,即当$x,y\in V$时,有唯一的$z=x+y\in V$,且此加法运算满足下面4条性质:
            \begin{enumerate}[1)]
                \item $x+y=y+x$(交换律)
                \item $x+(y+z)=(x+y)+z$(结合律)
                \item 存在{\heiti 零元素$\boldsymbol{0}\in V$}:对$V$中任一元素$x$都有$x+\boldsymbol{0}=x$
                \item 存在{\heiti 负元素}:对任一元素$x\in V$,存在一个元素$y\in V$,使得$x+y=\boldsymbol{0}$,称$y$为$x$的负元素(或相反元素),记为$-x$,即$x+(-x)=\boldsymbol{0}$
            \end{enumerate}
            \item 在$V$中定义一个封闭的数乘运算,即当$x\in V,\lambda\in K$时,有唯一的$\lambda x\in V$,且此数称运算满足下面4条性质:
            \begin{enumerate}[1)]
                \item $(\lambda+\mu)x=\lambda x+\mu x$(分配律)
                \item $\lambda(x+y)=\lambda x+\lambda y$(数因子分配律)
                \item $\lambda(\mu x)=(\lambda \mu )x$(结合律)
                \item $1\cdot x=x$
            \end{enumerate}
        \end{enumerate}
        其中$x,y,z$是$V$中的任意元素,$\lambda,\mu$是数域$K$中的任意数,1是数域$K$中的单位数。

        称$V$是数域$K$上的{\heiti 线性空间},记为$V(K)$.当$K$是实数域$\mathbf{R}$时,称$V$为{\heiti 实线性空间};
        当$K$是复数域$\mathbf{C}$时,称$V$为{\heiti 复线性空间}.
    \end{definition}

    \begin{theorem}
        线性空间的性质:\\
        {\heiti 性质1}$\quad$线性空间的零元素是唯一的.\\
        {\heiti 性质2}$\quad$线性空间中任一元素的负元是唯一的.\\
        {\heiti 性质3}$\quad$设$0,1,-1,\lambda\in K,x,-x,\boldsymbol{0}\in V$则有
            (1) $0x=\boldsymbol{0}$
            (2) $(-1)x=-x$
            (3) $\lambda\boldsymbol{0}=\boldsymbol{0}$
            (4) 若$\lambda x=\boldsymbol{0}$,则$\lambda=0$或$x=\boldsymbol{0}$\\
        {\heiti 性质4}$\quad$只含一个元素的线性空间称为{\heiti 零空间},显见,仅有的那个元素是零元素.
    \end{theorem}

\subsection{线性空间的基、维数与坐标}
    \begin{definition}[线性相关与线性无关]
        已知$V(K)$是线性空间,$\boldsymbol{\alpha}_1,\boldsymbol{\alpha}_2,\cdots,\boldsymbol{\alpha}_m$是$V(K)$的一组向量,
        如果存在一组不全为零的数$k_1,k_2,\cdots,k_m$使得$\displaystyle{\sum_{i=1}^m k_i\boldsymbol{\alpha}_i=\boldsymbol{0}}$,
        则称该向量组{\heiti 线性相关},否则称为{\heiti 线性无关}.
    \end{definition}

    \begin{definition}[维数]
        设$V$是数域$K$上的线性空间,
        \begin{enumerate}[(1)]
            \item 如果在$V$中可以找到任意多各线性无关的向量,则称$V$是{\heiti 无限维线性空间};
            \item 如果存在有限多个向量$\boldsymbol{\alpha}_1,\boldsymbol{\alpha}_2,\cdots,\boldsymbol{\alpha}_n(n\geq 1)\in V$,满足:
            \begin{enumerate}[1)]
                \item $\boldsymbol{\alpha}_1,\boldsymbol{\alpha}_2,\cdots,\boldsymbol{\alpha}_n$线性无关;
                \item $V$中任一向量都可由$\boldsymbol{\alpha}_1,\boldsymbol{\alpha}_2,\cdots,\boldsymbol{\alpha}_n$线性表出.
            \end{enumerate}
        \end{enumerate}
        则称$V$是{\heiti 有限维线性空间},称$\boldsymbol{\alpha}_1,\boldsymbol{\alpha}_2,\cdots,\boldsymbol{\alpha}_n$
        是$V$的一组{\heiti 基}(或{\heiti 基底}),$\boldsymbol{\alpha}_i$叫第$i$个{\heiti 基向量},基向量的个数$n$称为线性空间
        $V$的{\heiti 维数},记为$\mathrm{dim}(V)=n$,并称$V$是$n$维线性空间.
    \end{definition}

    \begin{theorem}
        设$\boldsymbol{\alpha}_1,\boldsymbol{\alpha}_2,\cdots,\boldsymbol{\alpha}_n$是$n$维线性空间$V$的一组基,对任意的
        $\boldsymbol{\alpha}\in V$,$\boldsymbol{\alpha}$可以唯一地由这一组基线性表出.
    \end{theorem}

    \begin{definition}[坐标]
        设$\boldsymbol{\alpha}_1,\boldsymbol{\alpha}_2,\cdots,\boldsymbol{\alpha}_n$是
        $n$维线性空间$V$的一组基,对任意的$\boldsymbol{\alpha}\in V$,若有一组有序数$x_1,x_2,\cdots,x_n$使得
        $\boldsymbol{\alpha}$可表示为
        $$\boldsymbol{\alpha}=x_1\boldsymbol{\alpha}_1+x_2\boldsymbol{\alpha}_2+\cdots+x_n\boldsymbol{\alpha}_n$$
        这组有序数就称为向量$\boldsymbol{\alpha}$在基$\boldsymbol{\alpha}_1,\boldsymbol{\alpha}_2,\cdots,\boldsymbol{\alpha}_n$下的
        {\heiti 坐标},记为
        $$\boldsymbol{x}=(x_1,x_2,\cdots,x_n)\quad \mbox{或} \quad \boldsymbol{x}={(x_1,x_2,\cdots,x_n)}^\mathrm{T}$$
    \end{definition}

    \begin{theorem}
        设$\boldsymbol{\alpha}_1,\boldsymbol{\alpha}_2,\cdots,\boldsymbol{\alpha}_n$与$\boldsymbol{\beta}_1,\boldsymbol{\beta}_2,\cdots,\boldsymbol{\beta}_n$
        是$n$维线性空间$V$的两组基,并且
        \begin{equation}\label{6.1}\tag{6.1}
        (\boldsymbol{\beta}_1\boldsymbol{\beta}_2\cdots\boldsymbol{\beta}_n)=(\boldsymbol{\alpha}_1\boldsymbol{\alpha}_2\cdots\boldsymbol{\alpha}_n)\boldsymbol{P}
        \end{equation}
        $$\boldsymbol{P}=\left(\begin{array}{cccc}
            a_{11} & a_{12} & \cdots & a_{1n}\\
            a_{21} & a_{22} & \cdots & a_{2n}\\
            \vdots & \vdots & &\vdots\\
            a_{n1} & a_{n2} & \cdots & a_{nn}
        \end{array}\right)$$
        若$V$中任意元素$\boldsymbol{\alpha}$在这两组基下的坐标分别是$(x_1,x_2,\cdots,x_n)$和$(y_1,y_2.\cdots,y_n)$,则
        $${(y_1,y_2,\cdots,y_n)}^\mathrm{T}=\boldsymbol{P}^\mathrm{-1}{(x_1,x_2,\cdots,x_n)}^\mathrm{T}$$

        $\boldsymbol{P}$称为从基底$\boldsymbol{\alpha}_1,\boldsymbol{\alpha}_2,\cdots,\boldsymbol{\alpha}_n$到基底$\boldsymbol{\beta}_1,\boldsymbol{\beta}_2,\cdots,\boldsymbol{\beta}_n$
        的{\heiti 过渡矩阵},称式\eqref{6.1}为{\heiti 基底变换公式},称式\eqref{6.2}为{\heiti 坐标变换公式}。其中式\eqref{6.2}为
        \begin{equation}\label{6.2}\tag{6.2}
            \left(\begin{array}{c}
                x_1\\
                x_2\\
                \vdots\\
                x_n
            \end{array}\right)=\boldsymbol{P}\left(\begin{array}{c}
                y_1\\
                y_2\\
                \vdots\\
                y_n
            \end{array}\right)
        \end{equation}
    \end{theorem}

\subsection{线性空间的子空间}
    \begin{definition}[子空间]
        设$W$是数域$K$上的线性空间$V$的一个非空子集,若$W$关于$V$上的加法和数乘也构成数域$K$上的一个线性空间,则称$W$是$V$的
        一个{\heiti 线性子空间},简称{\heiti 子空间},记为$W\subseteq V$,若$W\neq V$,记为$W\subset V$.
    \end{definition}

    \begin{theorem}
        线性空间$V$的一个非空子集$W$是$V$的子空间的充要条件是
        \begin{enumerate}[(1)]
            \item 对任意的$\boldsymbol{\alpha},\boldsymbol{\beta}\in W$,有$\boldsymbol{\alpha}+\boldsymbol{\beta}\in W$,即对加法封闭;
            \item 对任意的$\boldsymbol{\alpha}\in W,\lambda\in K$,有$\lambda\boldsymbol{\alpha}\in W$,即对数乘封闭.
        \end{enumerate}
        亦可把上面两个条件合并得到

        对任意的$\boldsymbol{\alpha},\boldsymbol{\beta}\in W,\lambda,\mu\in K$,有$\lambda\boldsymbol{\alpha}+\mu\boldsymbol{\beta}\in W$。
    \end{theorem}

    \begin{definition}
        每个线性空间至少有两个子空间,一个是仅有零向量构成的,称为{\heiti 零子空间};一个是它自身,称为{\heiti 平凡子空间},而其他
        的子空间称为非平凡子空间(或{\heiti 真子空间}).
    \end{definition}

    \begin{definition}
        设$V$是数域$K$上的线性空间,向量组$\boldsymbol{\alpha}_1,\boldsymbol{\alpha}_2,\cdots,\boldsymbol{\alpha}_s\in V$,由这组向量所
        有可能的线性组合构成的集合
        $$W(\boldsymbol{\alpha}_1,\boldsymbol{\alpha}_2,\cdots,\boldsymbol{\alpha}_s)=\left\{\boldsymbol{\alpha}:\boldsymbol{\alpha}=\sum_{i=1}^s k_i\boldsymbol{\alpha}_i,\quad k_i\in K,i=1,2,\cdots,s\right\}$$
        是非空的,容易验证$W$是$V$的子空间,这样的子空间称为由向量组$\boldsymbol{\alpha}_1,\boldsymbol{\alpha}_2,\cdots,\boldsymbol{\alpha}_s$生成的子空间,记作
        $$\mathrm{span}\{\boldsymbol{\alpha}_1,\boldsymbol{\alpha}_2,\cdots,\boldsymbol{\alpha}_s\} \quad \mbox{或} \quad L\{\boldsymbol{\alpha}_1,\boldsymbol{\alpha}_2,\cdots,\boldsymbol{\alpha}_s\}$$
        特别地,零子空间是由零向量生成的子空间$\mathrm{span}\{\boldsymbol{0}\}$

        设
        $$\left\{\begin{array}{l}
            a_{11}x_1+a_{12}x_2+\cdots+a_{1n}x_n=0,\\
            a_{21}x_1+a_{22}x_2+\cdots+a_{2n}x_n=0,\\
            \cdots\\
            a_{n1}x_1+a_{n2}x_2+\cdots+a_{nn}x_n=0,\\
        \end{array}\right.\quad (a_{ij}\in\mathbf{R})$$
        是实数域$\mathbf{R}$上的齐次线性方程组,它的全体解向量是$\mathbf{R}^n$中的一个非空子集$W$,显然$W$是$\mathbf{R}^n$
        的一个子空间,这个子空间称为该齐次方程组的{\heiti 解空间}.该齐次方程组的任一组基础解系是$W$的一组基,若方程组的系数矩阵的秩为
        $r$,则$\mathrm{dim}(W)=n-r$.
    \end{definition}

    \begin{definition}[子空间的交与和]
        设$W_1,W_2$是数域$K$上线性空间$V$的两个子空间,定义$W_1$与$W_2$的交为
        $$W_1\cap W_2=\{\boldsymbol{\alpha}:\boldsymbol{\alpha}\in W_1,\boldsymbol{\alpha}\in W_2\}$$
        $W_1$与$W_2$的和为
        $$W_1+W_2=\{\boldsymbol{\gamma}:\boldsymbol{\gamma}=\boldsymbol{\alpha}+\boldsymbol{\beta},\mbox{对所有的}\boldsymbol{\alpha}\in W_1,\boldsymbol{\beta}\in W_2\}$$
    \end{definition}

    \begin{theorem}
        数域$K$上线性空间$V$的两个子空间$W_1$与$W_2$的交与和仍是$V$的子空间.
    \end{theorem}

    \begin{theorem}
        若$W_1,W_2$是线性空间$V$的两个有限维子空间,则
        $$\mathrm{dim}W_1+\mathrm{dim}W_2=\mathrm{dim}(W_1+W_2)+\mathrm{dim}(W_1\cap W_2)$$
    \end{theorem}

    \begin{definition}[直和]
        若$W_1+W_2$中任一向量都只能唯一地表示为子空间$W_1$的一个向量与子空间$W_2$的一个向量的和,则称$W_1+W_2$是{\heiti 直和}(或{\heiti 直接和}),
        记为$W_1\oplus W_2$或$W_1+W_2$。
        
        若$W=W_1\oplus W_2$,则称在$W$内$W_1$是$W_2$的{\heiti 补空间},或$W_2$是$W_1$的{\heiti 补空间}.
    \end{definition}

    \begin{theorem}
        $W_1+W_2$是直和的充要条件是$W_1\cap W_2=\{\boldsymbol{0}\}$。
    \end{theorem}

    \begin{theorem}
        $W_1+W_2$是直和的充要条件是$\mathrm{dim}(W_1+ W_2)=\mathrm{dim}(W_1)+\mathrm{dim}(W_2)$。
    \end{theorem}

    \begin{definition}
        设$W_1,W_2.\cdots,W_m$是线性空间$V$的子空间,若
        \begin{enumerate}[(1)]
            \item $W_1+W_2+\cdots+W_m=V$;
            \item $W_1\cap W_2=\{\boldsymbol{0}\},(W_1+W_2)\cap W_3=\{\boldsymbol{0}\},\cdots,(W_1+W_2+\cdots+W_{m-1})\cap W_m=\{\boldsymbol{0}\}$
        \end{enumerate}
        则称$V$是$W_1,W_2,\cdots,W_m$的直和,记作
        $$V=W_1\oplus W_2\oplus \cdots \oplus W_m$$
    \end{definition}

\subsection{线性变换}
    \begin{definition}[线性变换]
        设$V_1,V_2$都是数域$K$上线性空间,根据某一规则$T$,对$V_1$中的任一元素$\boldsymbol{\alpha}$,有$V_2$中的唯一元素
        $\boldsymbol{\alpha}'$与之对应,即$T\boldsymbol{\alpha}=\boldsymbol{\alpha}'$,则称$T$为$V_1$到$V_2$的
        {\heiti 映射}。

        如果$V_1$到$V_2$的映射$T$还满足
        $$T(\boldsymbol{\alpha}+\boldsymbol{\beta})=T\boldsymbol{\alpha}+T\boldsymbol{\beta},\quad T(\lambda\boldsymbol{\alpha})=\lambda T\boldsymbol{\alpha}$$
        其中$\boldsymbol{\alpha},\boldsymbol{\beta}\in V_1,\lambda\in K$,则称$T$为$V_1$到$V_2$的{\heiti 线性映射}。在$V_1=V_2=V$时,称这个$T$为$V$上的
        {\heiti 线性变换}.
    \end{definition}

    \begin{definition}
        几个特殊的线性变换:
        \begin{enumerate}[(1)]
            \item {\heiti 数乘变换:}设$k$是数域$K$内的一个常数,对任意的$\boldsymbol{\alpha}\in V$,令$T_k\boldsymbol{\alpha}=k\boldsymbol{\alpha}$,容易验证
            $T_k$是$V$上的线性变换.
            \item {\heiti 恒等变换:}设$T\boldsymbol{\alpha}=\boldsymbol{\alpha}$,即$T$把$V$中的任意元素$\boldsymbol{\alpha}$变成自身,则称$T$为{\heiti 恒等变换}或{\heiti 单位变换},可记为$\boldsymbol{I}$或$\boldsymbol{E}$,即$\boldsymbol{I}\boldsymbol{\alpha}=\boldsymbol{\alpha}$.
            \item {\heiti 零变换:}设$T\boldsymbol{\alpha}=\boldsymbol{0}$,即$T$把$V$中的任意元素$\boldsymbol{\alpha}$变成零元素,则称$T$为零变换,记为$T_0$,即$T_0\boldsymbol{\alpha}=\boldsymbol{0}$
        \end{enumerate}
    \end{definition}

    \begin{definition}[像空间与核空间]
        (1)若$V_1,V_2$都是数域$K$上线性空间,设线性映射$T:V_1\to V_2$,则称$V_1$中所有元素的像的集合为{\heiti 像空间}记作$\mathcal{R} (T)$或$\mathrm{Im}(T)$, 简记为$\mathcal{R}$.
        
        (2)对于$V_1$到$V_2$的线性映射$T$,称集合
        $$N=N(T)=\{\boldsymbol{\alpha}:T\boldsymbol{\alpha}=\boldsymbol{0}',\boldsymbol{\alpha}\in V_1\}$$
        为$T$的{\heiti 核空间},也记作$\mathrm{ker}(T)$,其中$\boldsymbol{0}'$是$V_2$的零元。
    \end{definition}

    \begin{theorem}
        线性映射的像空间与核空间是线性子空间.
    \end{theorem}

    \begin{theorem}
        设对$V$的两个线性变换$T_1$和$T_2$,有
        $$T_i\boldsymbol{\varepsilon }_i=T_2\boldsymbol{\varepsilon }_i,\quad i=1,2,\cdots,n$$
        则$T_1=T_2$(这里两个线性变换相等是指它们对$V$的任一向量的像相等).
    \end{theorem}

    \begin{theorem}
        设$\boldsymbol{\varepsilon}_1,\boldsymbol{\varepsilon}_2,\cdots,\boldsymbol{\varepsilon}_n$是线性空间$V$的一组基,任给
        $\boldsymbol{\alpha}_1,\boldsymbol{\alpha}_2,\cdots,\boldsymbol{\alpha}_n\in V$,则一定存在唯一的线性变换$T$,使得
        $$T\boldsymbol{\varepsilon}_i=\boldsymbol{\alpha}_i,\quad i=1,2,\cdots,n$$
    \end{theorem}

    \begin{theorem}
        在线性空间$V$的一组基$\boldsymbol{\varepsilon}_1,\boldsymbol{\varepsilon}_2,\cdots,\boldsymbol{\varepsilon}_n$下,
        $V$上的线性变换$T$与$n$阶方阵$\boldsymbol{A}$一一对应,且它们的对应关系是$(T\boldsymbol{\varepsilon}_1T\boldsymbol{\varepsilon}_2\cdots T\boldsymbol{\varepsilon}_n)=(\boldsymbol{\varepsilon}_1\boldsymbol{\varepsilon}_2\cdots\boldsymbol{\varepsilon}_n)\boldsymbol{A}$.
        即$\boldsymbol{A}$的第$i$个列向量是$T\boldsymbol{\varepsilon}_i$在基$\boldsymbol{\varepsilon}_1,\boldsymbol{\varepsilon}_2,\cdots,\boldsymbol{\varepsilon}_n$下的坐标。
    \end{theorem}

    \begin{theorem}
        在$n$维线性空间$V$中取定两组基底$\boldsymbol{\varepsilon}_1,\boldsymbol{\varepsilon}_2,\cdots,\boldsymbol{\varepsilon}_n$与
        $\boldsymbol{\omega}_1,\boldsymbol{\omega}_2,\cdots,\boldsymbol{\omega}_n$,设由基底$\boldsymbol{\varepsilon}_1,\boldsymbol{\varepsilon}_2,\cdots,\boldsymbol{\varepsilon}_n$到
        $\boldsymbol{\omega}_1,\boldsymbol{\omega}_2,\cdots,\boldsymbol{\omega}_n$的过渡矩阵为
        $$\boldsymbol{P}=\left(\begin{array}{ccc}
            a_{11} & \cdots &a_{1n} \\
            \vdots & & \vdots\\
            a_{n1} & \cdots &a_{nn} 
        \end{array}\right)\quad(\boldsymbol{P}\mbox{可逆})$$
        即$(\boldsymbol{\omega}_1\boldsymbol{\omega}_2\cdots\boldsymbol{\omega}_n)=(\boldsymbol{\varepsilon}_1\boldsymbol{\varepsilon}_2\cdots\boldsymbol{\varepsilon}_n)\boldsymbol{P}$.
        并设$V$上的线性变换$T$在这两组基底下的矩阵分别是$\boldsymbol{A}$和$\boldsymbol{B}$,则
        $$\boldsymbol{B}=\boldsymbol{P}^{-1}\boldsymbol{A}\boldsymbol{P}$$
        即$\boldsymbol{A}$相似于$\boldsymbol{B}$($\boldsymbol{A}\sim \boldsymbol{B}$)
    \end{theorem}

    \begin{theorem}
        设$\boldsymbol{A},\boldsymbol{B}$都是$n$阶方阵,则$\boldsymbol{A}\sim \boldsymbol{B}$的充要条件是:
        它们是$n$维的线性空间$V$上的某个线性变换$T$在不同基底下的矩阵.
    \end{theorem}


\subsection{线性变换的特征值和特征向量}
    \begin{definition}[线性变换的特征值与特征向量]
        设$V$是数域$K$上线性空间,$T$是$V$上的一个线性变换,若对$K$中的一个数$\lambda$,存在$V$的一个非零向量
        $\xi$,使得$$T\boldsymbol{\xi}=\lambda\boldsymbol{\xi}$$
        则称$\lambda$是线性变换$T$的一个{\heiti 特征值},$\boldsymbol{\xi}$是$T$的属于$\lambda$的{\heiti 特征向量}.
    \end{definition}

    \begin{theorem}
        设$V$是数域$K$上线性空间,$T$是$V$上的一个线性变换,则$T$的属于特征值$\lambda$的
        所有特征向量和零向量一起构成了$V$的一个子空间,该子空间记为
        $$V_{\lambda}=\{\boldsymbol{\xi}\in V:T\boldsymbol{\xi}=\lambda\boldsymbol{\xi}\}$$
    \end{theorem}

    \begin{definition}[特征子空间]
        称$V_\lambda$为线性变换$T$对应于特征值$\lambda$的{\heiti 特征子空间}.
    \end{definition}

    \begin{theorem}
        计算线性变换$T$的特征值和特征向量的步骤:
        \begin{enumerate}[(1)]
            \item 给定$V$的一组基,求出$T$在这组基下的矩阵$A$;
            \item 计算特征多项式$p(\lambda)=|\lambda\boldsymbol{E}-\boldsymbol{A}|$
            \item 求$p(\lambda)=0$的含于$K$的根($T$的特征值)$$\lambda_1,\lambda_2,\cdots,\lambda_n$$
            \item 对每个$\lambda_i(i=1,2,\cdots,n)$,求齐次线性方程组$$(\lambda_i\boldsymbol{E}-\boldsymbol{A})\boldsymbol{x}=\boldsymbol{\theta}$$的一个基础解系;
            \item 以求出的基础解系为坐标写出$V$的一个向量组,它就是$V_{\lambda_i}(i=1,2,\cdots,n)$的一组基.
        \end{enumerate}

        若计算$\boldsymbol{A}$的特征值和特征向量,只需直接进行第(2)、(3)、(4)步,且第(3)步须求出
        $p(\lambda)=0$的全部复根(即$\boldsymbol{A}$的全部特征值).
    \end{theorem}

    \begin{definition}
        将线性变换$T$在任一组基下的矩阵的特征多项式称为$T$的{\heiti 特征多项式},而将$T$在任一组基下的矩阵的特征矩阵称为$T$的{\heiti 特征矩阵}.
    \end{definition}

    \begin{theorem}
        有限维线性空间上的线性变换的特征值和特征多项式与所选基底无关。
    \end{theorem}

    \begin{theorem}
        设$\lambda_1,\lambda_2,\cdots,\lambda_s$是线性变换$T$(或矩阵$\boldsymbol{A}$)的$s$个互异的特征值,$\boldsymbol{\xi}_i$是
        属于特征值$\lambda_i(i=1,2,\cdots,s)$的特征向量,则$\boldsymbol{\xi}_1,\boldsymbol{\xi}_2,\cdots,\boldsymbol{\xi}_s$线性无关.
    \end{theorem}

    \begin{definition}
        设$T$是$n$维线性空间$V$上的一个线性变换,若$T$在某组基$\boldsymbol{\varepsilon}_1,\boldsymbol{\varepsilon}_2,\cdots,\boldsymbol{\varepsilon}_n$
        下的矩阵为对角矩阵,则称$T$有{\heiti 最简表示}.
    \end{definition}

    \begin{theorem}
        $n$维线性空间$V$上的一个线性变换$T$有最简表示的充要条件是$T$有$n$个线性无关的特征向量.
    \end{theorem}

    \begin{theorem}
        若线性变换$T$有$n$个互异的特征值,则$T$有最简表示.
    \end{theorem}

    \begin{theorem}
        若$\lambda_1,\lambda_2$是线性变换$T$的两个不同的特征值,$\boldsymbol{\xi}_1,\boldsymbol{\xi}_2,\cdots,\boldsymbol{\xi}_s$是$T$
        的属于$\lambda_1$的线性无关的特征向量,$\boldsymbol{\nu}_1,\boldsymbol{\nu}_2,\cdots,\boldsymbol{\nu}_t$是$T$的属于$\lambda_2$的线性无关的特征向量,
        则$\boldsymbol{\xi}_1,\boldsymbol{\xi}_2,\cdots,\boldsymbol{\xi}_s;\boldsymbol{\nu}_1,\boldsymbol{\nu}_2,\cdots,\boldsymbol{\nu}_t$线性无关.
    \end{theorem}

    \begin{theorem}
        若$\lambda_0$是线性变换$T$的$s$重特征值,则$T$的属于$\lambda_0$的特征向量中,线性无关的最大组包含的向量的个数不超过$s$.
    \end{theorem}

    \begin{theorem}
        $n$维线性空间上的线性变换$T$有最简表示$\Longleftrightarrow $$T$有$n$个特征值(包括重数),且对$T$的每个$s_i$重特征值$\lambda_i$,
        其对应的特征矩阵$\lambda_i\boldsymbol{E}-\boldsymbol{A}$的秩为$n-s_i$. 
    \end{theorem}