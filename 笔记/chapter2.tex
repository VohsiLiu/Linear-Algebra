\section{矩阵,向量}

\subsection{矩阵和$n$维向量的概念}
\begin{definition}[矩阵]
    有$m\times n$个数$a_{ij}(i=1,2,\cdots,m;j=1,2,\cdots,n)$排成$m$行$n$列数表,外加括号,写成
    $$
    \left(\begin{array}{cccc}
    a_{11} & a_{12} & \cdots & a_{1 n} \\
    a_{21} & a_{22} & \cdots & a_{2 n} \\
    \vdots & \vdots & & \vdots \\
    a_{m 1} & a_{m 2} & \cdots & a_{m n}
    \end{array}\right)
    $$
    叫做$m$行$n$列矩阵,简称{\heiti $m\times n$矩阵}。
\end{definition}

\begin{definition}[$n$维向量]
    $n$个数$a_1,a_2\cdots,a_n$组成的有序数组
    $$(a_1,a_2,\cdots,a_n)\quad \mbox{或}  \quad   \left(\begin{array}{c}
        a_{1}  \\
        a_{2}  \\
        \vdots \\
        a_{n} 
        \end{array}\right)$$
    称为{\heiti $n$维向量},前者称为{\heiti 行向量},后者称为{\heiti 列向量}。
\end{definition}

\begin{remark}[]
    几个特殊的矩阵:
    
    设$\boldsymbol{A}=(a_{ij})_{n\times n}$为$n$阶方阵,如果当$i\neq j$时,$a_{ij}=0$,即
    $$
    \boldsymbol{A}=\left(\begin{array}{cccc}
    a_{11} & 0 & \cdots & 0 \\
    0 & a_{22} & \cdots & 0 \\
    \vdots & \vdots & & \vdots \\
    0 & \cdots & 0 & a_{n n}
    \end{array}\right)
    $$
    则称$\boldsymbol{A}$为{\heiti 对角矩阵}。为了书写简洁起见,也经常用$\mathrm{diag}(a_{11},a_{22},\cdots,a_{nn})$
    表示此对角矩阵$\boldsymbol{A}$。

    如果对角矩阵$\boldsymbol{A}$中的$a_{11},a_{22},\cdots,a_{nn}=k$,则称$\boldsymbol{A}$为{\heiti 数量矩阵},当$k=1$时,称为{\heiti 单位矩阵},记为
    $\boldsymbol{E}$或$\boldsymbol{I}$

    设$\boldsymbol{A}=(a_{ij})_{n\times n}$为$n$阶方阵,如果当$i>j$时,$a_{ij}=0$,即
    $$
    \boldsymbol{A}=\left(\begin{array}{cccc}
    a_{11} & a_{12} & \cdots & a_{1 n} \\
    0 & a_{22} & \cdots & a_{2 n} \\
    \vdots & \vdots & & \vdots \\
    0 & 0 & \cdots & a_{n n}
    \end{array}\right)
    $$
    则称$\boldsymbol{A}$为{\heiti 上三角形矩阵}。类似地,可定义{\heiti 下三角形矩阵}
    $$
    \boldsymbol{B}=\left(\begin{array}{cccc}
    b_{11} & 0 & \cdots & 0 \\
    b_{21} & b_{22} & \cdots & 0\\
    \vdots & \vdots & & \vdots \\
    b_{n1} & b_{n2} & \cdots & b_{n n}
    \end{array}\right)
    $$

    设$\boldsymbol{A}=(a_{ij})_{n\times n}$为$n$阶方阵,如果矩阵的元素满足$a_{ij}=a_{ji}(i,j=1,2,\cdots,n)$,则称$\boldsymbol{A}$为
    {\heiti 对称矩阵};如果矩阵的元素满足$a_{ij}=-a_{ji}(i,j=1,2,\cdots,n)$,则称$\boldsymbol{A}$为
    {\heiti 反对称矩阵}。例如
    $$
    \boldsymbol{A}=\left(\begin{array}{ccc}
    1 & 0 & -1 \\
    0 & 2 & 5 \\
    -1 & 5 & 3
    \end{array}\right), \quad \boldsymbol{B}=\left(\begin{array}{ccc}
    0 & 1 & -6 \\
    -1 & 0 & -7 \\
    6 & 7 & 0
    \end{array}\right)
    $$
    依次是对称矩阵和反对称矩阵。
\end{remark}

\begin{definition}[转置矩阵]
    把$m\times n$矩阵
    $$
    \boldsymbol{A}=\left(\begin{array}{cccc}
    a_{11} & a_{12} & \cdots & a_{1 n} \\
    a_{21} & a_{22} & \cdots & a_{2 n} \\
    \vdots & \vdots & & \vdots \\
    a_{m 1} & a_{m 2} & \cdots & a_{m n}
    \end{array}\right)
    $$
    的行与列互换后得到一个$n\times m$矩阵,称为$\boldsymbol{A}$的{\heiti 转置矩阵},记为$A^\mathrm{T}$或$A^{\prime}$。
\end{definition}

\begin{definition}[方阵的行列式]
    行列式
    $$
    \left|\begin{array}{cccc}
    a_{11} & a_{12} & \cdots & a_{1 n} \\
    a_{21} & a_{22} & \cdots & a_{2 n} \\
    \vdots & \vdots & & \vdots \\
    a_{n 1} & a_{n 2} & \cdots & a_{n n}
    \end{array}\right|
    $$
    称为方阵$\boldsymbol{A}=(a_{ij})_{n\times n}$的行列式,记为$|\boldsymbol{A}|$。如果$|\boldsymbol{A}|\neq 0$,则称矩阵$\boldsymbol{A}$
    是{\heiti 非异矩阵},如果$|\boldsymbol{A}|=0$,则称矩阵$\boldsymbol{A}$是{\heiti 奇异矩阵}或{\heiti 退化矩阵}。
\end{definition}

\subsection{矩阵运算}
\begin{definition}[矩阵的加法]
    设$\boldsymbol{A}=(a_{ij})_{m\times n}$与$\boldsymbol{B}=(b_{ij})_{m\times n}$是两个同型矩阵,矩阵$\boldsymbol{A}$和矩阵$\boldsymbol{B}$
    的和定义为${(a_{ij}+b_{ij})}_{m\times n}$,记为$\boldsymbol{A}+\boldsymbol{B}$:
    $$\boldsymbol{A}+\boldsymbol{B}={(a_{ij}+b_{ij})}_{m\times n}$$
\end{definition}

\begin{theorem}
    矩阵加法的性质:
    \begin{enumerate}
        \item 交换律:$\boldsymbol{A}+\boldsymbol{B}=\boldsymbol{B}+\boldsymbol{A}$
        \item 结合律:$(\boldsymbol{A}+\boldsymbol{B})+\boldsymbol{C}=\boldsymbol{A}+(\boldsymbol{B}+\boldsymbol{C})$
        \item 零矩阵:对任一矩阵$\boldsymbol{A}$,$\boldsymbol{A}+\boldsymbol{O}=\boldsymbol{A}=\boldsymbol{O}+\boldsymbol{A}$($\boldsymbol{O}$与$\boldsymbol{A}$是同型矩阵)
        \item 负矩阵:对任一矩阵$\boldsymbol{A}=(a_{ij})$,可定义$-\boldsymbol{A}=(-a_{ij})$,称$-\boldsymbol{A}$为$\boldsymbol{A}$的负矩阵。显然$\boldsymbol{A}+(-\boldsymbol{A})=\boldsymbol{O}$
    \end{enumerate}
\end{theorem}

\begin{definition}[矩阵的数乘]
    数$k$与矩阵$\boldsymbol{A}={(a_{ij})}_{m\times n}$相乘的积的定义为${(ka_{ij})}_{m\times n}$,记为$k\boldsymbol{A}$:
    $$k\boldsymbol{A}={(ka_{ij})}_{m\times n}$$
\end{definition}

\begin{theorem}[方阵的数乘与其行列式]
    若$\boldsymbol{A}$是$n$阶方阵,$k$为任意数,则有$|k\boldsymbol{A}|=k^n|\boldsymbol{A}|$
\end{theorem}

\begin{definition}[矩阵的乘法]
    设$\boldsymbol{A}={(a_{ij})}_{m\times l},\boldsymbol{B}={(b_{ij})}_{l\times n}$,则$\boldsymbol{A}$与$\boldsymbol{B}$的乘积$\boldsymbol{A}\boldsymbol{B}$定义为
    $\boldsymbol{C}={(c_{ij})}_{m\times n}$,其中
    $$c_{ij}=\sum _{k=1}^l a_{ik}b_{kj}=a_{i1}b_{1j}+a_{i2}b_{2j}+\cdots+a_{il}b_{lj}\quad (i=1,2,\cdots,m;j=1,2,\cdots,n)$$
\end{definition}

\begin{remark}
    只有当$\boldsymbol{A}$的列数和$\boldsymbol{B}$的行数相等时,乘积$\boldsymbol{A}\boldsymbol{B}$才有意义,且乘积矩阵$\boldsymbol{C}$的行数与矩阵$\boldsymbol{A}$的行数相同,
    $\boldsymbol{C}$的列数与矩阵$\boldsymbol{B}$的列数相同。

    矩阵乘法不满足交换律和消去律,但满足结合律,数乘结合律和分配律。
\end{remark}

\begin{definition}[矩阵多项式]
    设$\boldsymbol{A}$是$n$阶方阵,定义$\boldsymbol{A}^0=\boldsymbol{E}_n,\boldsymbol{A}^1=\boldsymbol{A},\boldsymbol{A}^2=\boldsymbol{A}\boldsymbol{A},\cdots,\boldsymbol{A}^{k+1}=\boldsymbol{A}^k\boldsymbol{A}$
    其中$k$为正整数,这也就是说$\boldsymbol{A}^k$是$k$个$\boldsymbol{A}$连乘,称$\boldsymbol{A}^k$是方阵$\boldsymbol{A}$的$k${\heiti 次幂}。再任取$m+1$个实数$a_0,a_1,\cdots,a_m$,显然
    $$f(\boldsymbol{A})=a_0\boldsymbol{E}+a_1\boldsymbol{A}+\cdots+a_m\boldsymbol{A}^m$$
    仍为$n$阶矩阵。称$f(\boldsymbol{A})$为{\heiti 矩阵多项式}
\end{definition}

\begin{theorem}
    转置矩阵的性质:
    \begin{enumerate}
        \item ${({\boldsymbol{A}}^\mathrm{T})}^\mathrm{T}=\boldsymbol{A};\quad |\boldsymbol{A}^\mathrm{T}|=|\boldsymbol{A}|$
        \item ${(\boldsymbol{A}+\boldsymbol{B})}^\mathrm{T}=\boldsymbol{A}^\mathrm{T}+\boldsymbol{B}^\mathrm{T}$
        \item ${(k\boldsymbol{A})}^\mathrm{T}=k\boldsymbol{A}^\mathrm{T}$
        \item ${(\boldsymbol{A}\boldsymbol{B})}^\mathrm{T}=\boldsymbol{B}^\mathrm{T}\boldsymbol{A}^\mathrm{T}$,该式还可以推广为${(\boldsymbol{A}_1\boldsymbol{A}_2\cdots\boldsymbol{A}_k)}^\mathrm{T}=\boldsymbol{A}_k^\mathrm{T}\cdots\boldsymbol{A}_2^\mathrm{T}\boldsymbol{A}_1^\mathrm{T}$
    \end{enumerate}
\end{theorem}

\subsection{分块矩阵}
\begin{definition}[分块矩阵]
    对于一个$m\times n$矩阵,如果在行的方向分成$s$块,在列的方向分成$t$块,就得到$\boldsymbol{A}$的一个$s\times t$分块矩阵,记作
    $$
    \begin{array}{c}
    \boldsymbol{A}=\left(\boldsymbol{A}_{k l}\right)_{s \times t}=\left(\begin{array}{cccc}
    \boldsymbol{A}_{11} & \boldsymbol{A}_{12} & \cdots & \boldsymbol{A}_{1 t} \\
    \boldsymbol{A}_{21} & \boldsymbol{A}_{22} & \cdots & \boldsymbol{A}_{2 t} \\
    \vdots & \vdots & & \vdots \\
    \boldsymbol{A}_{s 1} & \boldsymbol{A}_{s 2} & \cdots & \boldsymbol{A}_{s t}
    \end{array}\right) \quad \begin{array}{c}
    m_{1}\mbox{行} \\
    m_{2}\mbox{行} \\
    \vdots \\
    m_{s}\mbox{行}
    \end{array} \\
    \begin{array}{rrrrr}
    {\color{white} text}&n_{1}\mbox{列} & n_{2}\mbox{列} & \cdots & n_{t}\mbox{列}
    \end{array}
    \end{array}$$
    其中$m_1+m_2+\cdots+m_s=m,n_1+n_2+\cdots+n_t=n$,而$\boldsymbol{A}_{kl}(k=1,2,\cdots,s;l=1,2,\cdots,t)$称为$\boldsymbol{A}$的子块。
\end{definition}

\begin{theorem}
    分块矩阵的运算:
    \begin{enumerate}
        \item 分块矩阵的加法。设分块矩阵$\boldsymbol{A}={(\boldsymbol{A}_{k l})}_{s\times t},\boldsymbol{B}={(\boldsymbol{B}_{k l})}_{s\times t}$
        如果$\boldsymbol{A}$与$\boldsymbol{B}$对应的子块$\boldsymbol{A}_{kl}$和$\boldsymbol{B}_{kl}$都是同型矩阵,则
        $$\boldsymbol{A}+\boldsymbol{B}={(\boldsymbol{A}_{k l}+\boldsymbol{B}_{kl})}_{s\times t}$$
        \item 分块矩阵的数乘。设分块矩阵$\boldsymbol{A}={(\boldsymbol{A}_{k l})}_{s\times t}$,$k$是数,则
        $$k\boldsymbol{A}={(k\boldsymbol{A}_{k l})}_{s\times t}$$
        \item 分块矩阵的乘法。设$\boldsymbol{A}={(a_{ij})}_{m\times n},\boldsymbol{B}={(b_{ij})}_{n\times p}$
        如果把$\boldsymbol{A},\boldsymbol{B}$分别分块为$r\times s$和$s\times t$分块矩阵,且$\boldsymbol{A}$的列的分块法与$\boldsymbol{B}$的行的分块法完全相同,则
        $$
        \boldsymbol{A} \boldsymbol{B}=\left(\begin{array}{cccc}
        \boldsymbol{A}_{11} & \boldsymbol{A}_{12} & \cdots & \boldsymbol{A}_{1 s} \\
        \boldsymbol{A}_{21} & \boldsymbol{A}_{22} & \cdots & \boldsymbol{A}_{2 s} \\
        \vdots & \vdots & & \vdots \\
        \boldsymbol{A}_{r 1} & \boldsymbol{A}_{r 2} & \cdots & \boldsymbol{A}_{r s}
        \end{array}\right)\left(\begin{array}{cccc}
        \boldsymbol{B}_{11} & \boldsymbol{B}_{12} & \cdots & \boldsymbol{B}_{1 t} \\
        \boldsymbol{B}_{21} & \boldsymbol{B}_{22} & \cdots & \boldsymbol{B}_{2 t} \\
        \vdots & \vdots & & \vdots \\
        \boldsymbol{B}_{s 1} & \boldsymbol{B}_{s 2} & \cdots & \boldsymbol{B}_{s t}
        \end{array}\right)=\boldsymbol{C}=\left(\boldsymbol{C}_{k l}\right)_{r \times t}
        $$
        其中$\boldsymbol{C}$是$r\times t$分块矩阵,且
        $$\boldsymbol{C}_{kl}=\boldsymbol{A}_{k1}\boldsymbol{B}_{1l}+\boldsymbol{A}_{k2}\boldsymbol{B}_{2l}+\cdots+\boldsymbol{A}_{ks}\boldsymbol{B}_{sl}=\sum _{i=1}^s \boldsymbol{A}_{ki}\boldsymbol{B}_{il}(k=1,2,\cdots,r;l=1,2,\cdots,t)$$
        \item 分块矩阵的转置。$s\times t$分块矩阵$\boldsymbol{A}={(\boldsymbol{A}_{k l})}_{s\times t}$的转置矩阵$\boldsymbol{A}^\mathrm{T}$为$t\times s$分块矩阵,如果记
        $\boldsymbol{A}^\mathrm{T}={(\boldsymbol{B}_{lk})}_{t\times s}$,则$$\boldsymbol{B}_{lk}=\boldsymbol{A}^\mathrm{T}_{kl}(l=1,2,\cdots,t;k=1,2,\cdots,s)$$
        \item 分块对角矩阵。设$n$阶矩阵$\boldsymbol{A}$的分块矩阵只有在对角线上有非零子块,其余子块都为零矩阵,且在对角线上的子块都是方阵,即
        $$
        \boldsymbol{A}=\left(\begin{array}{cccc}
        \boldsymbol{A}_{1} & & & \\
        & \boldsymbol{A}_{2} & & \\
        & & \ddots & \\
        & & & \boldsymbol{A}_{s}
        \end{array}\right)
        $$
        其中$\boldsymbol{A}_i(i=1,2,\cdots.s)$都是方阵,则称$\boldsymbol{A}$为分块对角矩阵,也称{\heiti 准对角矩阵}。
    \end{enumerate}    
\end{theorem}

\subsection{初等变换与初等矩阵}
\begin{definition}[初等变换]
    下面三种对矩阵的变换,统称为矩阵的初等变换:
    \begin{enumerate}
        \item 对调变换:互换矩阵$i,j$两行(列),记为$r_i\leftrightarrow r_j(c_i\leftrightarrow c_j)$
        \item 数乘变换:用任意数$k\neq 0$去乘矩阵的第$i$行(列),记为$kr_i(kc_i)$
        \item 倍加变换:把矩阵的第$i$行(列)的$k$倍加到第$j$行(列),其中$k$为任意数,记为$r_j+kr_i(c_j+kc_i)$
    \end{enumerate}
\end{definition}

\begin{definition}[初等矩阵]
    将单位矩阵$\boldsymbol{E}$做一次初等变换得到的矩阵称为初等矩阵。
    \begin{enumerate}
        \item 初等对调矩阵
        $$
        \begin{array}{c}
        \boldsymbol{E}_{(i,j)}=\left(\begin{array}{ccccccc}
        1 & & & & & & \\
        & \ddots & & & & &\\
        & & 0 & & 1& &\\
        & & & \ddots & & &\\
        & & 1&  &0 & &\\
        & & &  & & \ddots&\\
        & & &  & & &1
        \end{array}\right)\quad\begin{array}{c}     
             \\
             \\
           \leftarrow \mbox{第}i\mbox{行}  \\
            \\
            \leftarrow \mbox{第}j\mbox{行}  \\
            \\
            \\
           \end{array}\\
           \begin{array}{ccccccc}
            & &\uparrow  & \uparrow&    &   &{\color{white} 0}\\
            & & \mbox{第}i\mbox{列} & \mbox{第}j\mbox{列} & {\color{white} 0}& 
            \end{array}
        \end{array}
        $$
        即是将单位矩阵的第$i$行与第$j$行对调后所得的矩阵。
        \item 初等倍乘矩阵
        $$
        \begin{array}{c}
        \boldsymbol{E}(i(k))=\left(\begin{array}{ccccc}
        1 & & & &  \\
        & \ddots & & & \\
        & & k & & \\
        & & & \ddots & \\
        & & &  &1 \\
        \end{array}\right)\quad\begin{array}{c}     
             \\
             \\
           \leftarrow \mbox{第}i\mbox{行}  \\
            \\
            \\
          \end{array}\\
           \begin{array}{ccccccc}
            & &\uparrow  &&    &   &\\
            & & \mbox{第}i\mbox{列} &  & & 
            \end{array}
        \end{array}$$
        其中$k\neq 0$是任意数。既是将单位矩阵的第$i$个$1$换成$k$后所得的矩阵。
        \item 初等倍加矩阵
        $$
        \begin{array}{c}
        \boldsymbol{E}(i,j(k))=\left(\begin{array}{ccccccc}
        1 & & & & & & \\
        & \ddots & & & & &\\
        & & 1 &\cdots & k& &\\
        & & & \ddots & & &\\
        & & &  &1 & &\\
        & & &  & & \ddots&\\
        & & &  & & &1
        \end{array}\right)\quad\begin{array}{c}     
             \\
             \\
           \leftarrow \mbox{第}i\mbox{行}  \\
            \\
            \leftarrow \mbox{第}j\mbox{行}  \\
            \\
            \\
           \end{array}\\
           \begin{array}{cccccccc}
        {\color{white} 00} &{\color{white} 00} &\uparrow & & \uparrow&    &   &{\color{white} 0}\\
           {\color{white} 00}&{\color{white} 00} & \mbox{第}i\mbox{列}& & \mbox{第}j\mbox{列} & {\color{white} 0}& 
            \end{array}
        \end{array}
        $$
        即是将单位矩阵的第$i$行第$j$列的元素换成$k$后所得的矩阵。
    \end{enumerate}
\end{definition}

\begin{theorem}[初等变换与初等矩阵]
    设$\boldsymbol{A}$是一个$m\times n$矩阵,对$\boldsymbol{A}$施行一次初等行变换,相当于在$\boldsymbol{A}$的左边乘以一个
    相应的$m$阶初等矩阵;对$\boldsymbol{A}$施行一次初等行变换,相当于在$\boldsymbol{A}$的右边乘以一个相应的$n$阶初等矩阵。
\end{theorem}
\begin{remark}
    定理说法中“相应”的含义,具体来说,就是
    \begin{itemize}
        \item $\boldsymbol{E}(i,j)\boldsymbol{A}$:表示$\boldsymbol{A}$的第$i$行与第$j$行互换;
        \item $\boldsymbol{E}(i(k))\boldsymbol{A}$:表示$\boldsymbol{A}$的第$i$行乘以$k$;
        \item $\boldsymbol{E}(i,j(k))\boldsymbol{A}$:表示$\boldsymbol{A}$的第$j$行乘以$k$加到第$i$行上;
        \item $\boldsymbol{A}\boldsymbol{E}(i,j)$:表示$\boldsymbol{A}$的第$i$列与第$j$列互换;
        \item $\boldsymbol{A}\boldsymbol{E}(i(k))$:表示$\boldsymbol{A}$的第$i$列乘以$k$;
        \item $\boldsymbol{A}\boldsymbol{E}(i,j(k))$:表示$\boldsymbol{A}$的第$i$列乘以$k$加到第$j$列上.
    \end{itemize}
\end{remark}

\begin{definition}[行(列)等价矩阵,等价矩阵]
    如果矩阵$\boldsymbol{A}$经过有限次初行等变换变成矩阵$\boldsymbol{B}$,则称矩阵$\boldsymbol{A}$与$\boldsymbol{B}$
    {\heiti 行等价},记作$\boldsymbol{A}\stackrel{r}{\longrightarrow}\boldsymbol{B}$;
    如果矩阵$\boldsymbol{A}$经过有限次初等列变换变成矩阵$\boldsymbol{B}$,则称矩阵$\boldsymbol{A}$与$\boldsymbol{B}$
    {\heiti 列等价},记作$\boldsymbol{A}\stackrel{c}{\longrightarrow}\boldsymbol{B}$;
    如果矩阵$\boldsymbol{A}$经过有限次初等变换变成矩阵$\boldsymbol{B}$,则称矩阵$\boldsymbol{A}$与$\boldsymbol{B}$
    {\heiti 等价},记作$\boldsymbol{A}\stackrel{}{\longrightarrow}\boldsymbol{B}$。
\end{definition}

\begin{theorem}
    具有行等价关系的矩阵所对应的线性方程组有相同的解。
\end{theorem}

\begin{definition}[梯形矩阵]
    若矩阵$\boldsymbol{A}$满足下面两个条件:
    \begin{enumerate}[(1)]
        \item 若有零行,则零行全部在下方;
        \item 从第一行起,每行第一个非零元素前面的零的个数逐行增加。
    \end{enumerate}
    则称矩阵$\boldsymbol{A}$为{\heiti 行梯形矩阵}。若$\boldsymbol{A}$还满足:
    \begin{enumerate}[(3)]
        \item 非零行的第一个非零元素为$1$,且"$1$"所在列的其余元素全为零,则称$\boldsymbol{A}$为
        {\heiti 行简化梯形矩阵}。
    \end{enumerate}

    类似可定义{\heiti 列梯形矩阵}与{\heiti 列简化梯形矩阵}。
\end{definition}

\begin{theorem}[矩阵的标准型]
    若矩阵$\boldsymbol{A}$既是行简化矩阵,又是列简化矩阵,则称$\boldsymbol{A}$是{\heiti 标准形矩阵}。矩阵的标准型可写为
    $$\left(\begin{array}{cc}
        \boldsymbol{E} & \boldsymbol{O}\\
        \boldsymbol{O} & \boldsymbol{O}
    \end{array}\right)$$
    即左上角为单位矩阵,其余都是零矩阵。
\end{theorem}

\begin{theorem}[矩阵的化简]
    设$\boldsymbol{A}$为$m\times n$矩阵,则
    \begin{enumerate}[(1)]
        \item 存在$m$阶初等矩阵$\boldsymbol{P}_1,\boldsymbol{P}_2,\cdots,\boldsymbol{P}_s$使$\boldsymbol{P}_s\boldsymbol{P}_{s-1}\cdots\boldsymbol{P}_2\boldsymbol{P}_1\boldsymbol{A}$(即
        对$\boldsymbol{A}$施加有限次的初等行变换)成为$m\times n$阶行简化梯形矩阵。

        也存在$n$阶初等矩阵$\boldsymbol{Q}_1,\boldsymbol{Q}_2,\cdots,\boldsymbol{Q}_t$使$\boldsymbol{A}\boldsymbol{Q}_1\boldsymbol{Q}_2\cdots\boldsymbol{Q}_{t-1}\boldsymbol{Q}_t$(即
        对$\boldsymbol{A}$施加有限次的初等行变换)成为$m\times n$阶列简化梯形矩阵。
        \item 可以经过有限次初等行变换和初等列变换,将矩阵$\boldsymbol{A}$化为标准型。
    \end{enumerate}
\end{theorem}

\subsection{矩阵的秩}
\begin{definition}[矩阵的子式]
    设$\boldsymbol{A}$是一个$m\times n$矩阵,任取$\boldsymbol{A}$的$k$行和$k$列($0<k\leq\min\{m,n\}$),位于这些行列交叉处的
    $k^2$个元素,按原来的顺序排成的$k$阶行列式,称为矩阵$\boldsymbol{A}$的{\heiti $k$阶子式}。
\end{definition}

\begin{remark}
    例如,在$2\times 3$的矩阵
    $$\boldsymbol{A}=\left(\begin{array}{ccc}
        1 & 2 & -1 \\
        2 & -3 & 1
    \end{array}\right)$$
    中,1阶子式是由其中一个元素所构成,共有6个1阶子式,它的3个2阶子式是:
    $$\left|\begin{array}{ccc}
        1 & 2  \\
        2 & -3 
    \end{array}\right|,\quad
    \left|\begin{array}{ccc}
        1 & -1  \\
        2 & 1 
    \end{array}\right|,\quad
    \left|\begin{array}{ccc}
        2 & -1  \\
        -3 & 1 
    \end{array}\right|,\quad
    $$

    一般地,$m\times n$矩阵$\boldsymbol{A}$的$k$阶子式共有$\boldsymbol{C}_m^k\boldsymbol{C}_n^k$个。
\end{remark}

\begin{definition}[矩阵的秩]
    设$\boldsymbol{A}$是一个$m\times n$矩阵,如果$\boldsymbol{A}$中至少存在一个非零的$r$阶子式$D$,且所有
    $r+1$阶子式(如果存在的话)全为零,则称$D$为矩阵$\boldsymbol{A}$的最高阶非零子式,数$r$称为矩阵$\boldsymbol{A}$的
    {\heiti 秩},记为${\mathrm{rank}\boldsymbol{A}=r}$(或$\mathrm{r(\boldsymbol{A})}=r$)。并规定零矩阵的秩等于0.
\end{definition}

\begin{remark}
    与上述概念有关的结论为:
    \begin{enumerate}[(1)]
        \item $\mathrm{r}\boldsymbol{A}$是$\boldsymbol{A}$的非零子式的最高阶数;
        \item $0\leq \mathrm{r}(\boldsymbol{A}_{m\times n})\leq \min(m,n)$;
        \item $\mathrm{r}(\boldsymbol{A}^\mathrm{T})=\mathrm{r}(\boldsymbol{A})$;
        \item 对于$n$阶方阵$\boldsymbol{A}$,有$\mathrm{r}(\boldsymbol{A})=n$(即$\boldsymbol{A}$为满秩矩阵)$\Leftrightarrow |\boldsymbol{A}| \neq 0 $ $\Leftrightarrow \boldsymbol{A}$非异。
    \end{enumerate}
\end{remark}

\begin{theorem}
    初等行、列变换不改变矩阵的秩。
\end{theorem}

\begin{theorem}
    行梯形矩阵的秩等于它的非零行的行数。
\end{theorem}

\begin{theorem}
    任一满秩矩阵都可以经过若干次初等行变换为单位矩阵,也可以经过若干次初等列变换为单位矩阵。
\end{theorem}

\begin{theorem}[求矩阵的秩的方法]
    先用初等行、列变换将矩阵化为梯形矩阵,其非零行的行数就是矩阵的秩。
\end{theorem}

\subsection{可逆矩阵与伴随矩阵}
\begin{definition}[逆矩阵]
    对于$n$阶方阵$\boldsymbol{A}$,如果存在同阶方阵$\boldsymbol{B}$,使得
    $$\boldsymbol{A}\boldsymbol{B}=\boldsymbol{B}\boldsymbol{A}=\boldsymbol{E}$$
    则称矩阵$\boldsymbol{A}$是{\heiti 可逆矩阵},并称$\boldsymbol{B}$是矩阵$\boldsymbol{A}$的{\heiti 逆矩阵},记为$\boldsymbol{A}^{-1}$
\end{definition}


\begin{theorem}
    可逆矩阵的性质:
    \begin{enumerate}[(1)]
        \item 可逆矩阵的逆矩阵是唯一的。
        \item 若$\boldsymbol{A}$可逆,则$\boldsymbol{A}^{-1}$也可逆,且${(\boldsymbol{A}^{-1})}^{-1}=\boldsymbol{A}$;还有$|\boldsymbol{A}^{-1}|=|\boldsymbol{A}|^{-1}$
        \item 若$\boldsymbol{A}$可逆,数$k\neq 0$,则$k\boldsymbol{A}$可逆,且${(k\boldsymbol{A})}^{-1}=k^{-1}\boldsymbol{A}^{-1}$
        \item 若$\boldsymbol{A}$可逆,则$\boldsymbol{A}^\mathrm{T}$也可逆,且${(\boldsymbol{A}^\mathrm{T})}^{-1}={(\boldsymbol{A}^{-1})}^{\mathrm{T}}$
        \item 若$\boldsymbol{A}$、$\boldsymbol{B}$为同阶可逆矩阵,则$\boldsymbol{A}\boldsymbol{B}$也可逆,且${(\boldsymbol{A}\boldsymbol{B})}^{-1}=\boldsymbol{B}^{-1}\boldsymbol{A}^{-1}$
    \end{enumerate}
\end{theorem}

\begin{definition}[方阵的伴随矩阵]
    设$\boldsymbol{A}=(a_{ij})$为$n$阶方阵,$A_{ij}$是$|\boldsymbol{A}|$中元素$a_{ij}$的代数余子式,则称矩阵
    $$\boldsymbol{A}^{\ast }=\left(\begin{array}{cccc}
        \boldsymbol{A}_{11} &  \boldsymbol{A}_{12} & \cdots & \boldsymbol{A}_{1n}\\
        \boldsymbol{A}_{21} &  \boldsymbol{A}_{22} & \cdots & \boldsymbol{A}_{2n}\\
        \ddots &  \ddots &  & \ddots\\
        \boldsymbol{A}_{n1} &  \boldsymbol{A}_{n2} & \cdots & \boldsymbol{A}_{nn}\\
    \end{array}\right)^\mathrm{T}=\left(\begin{array}{cccc}
        \boldsymbol{A}_{11} &  \boldsymbol{A}_{21} & \cdots & \boldsymbol{A}_{n1}\\
        \boldsymbol{A}_{12} &  \boldsymbol{A}_{22} & \cdots & \boldsymbol{A}_{n2}\\
        \ddots &  \ddots &  & \ddots\\
        \boldsymbol{A}_{1n} &  \boldsymbol{A}_{2n} & \cdots & \boldsymbol{A}_{nn}\\
    \end{array}\right)$$
    为$\boldsymbol{A}$的{\heiti 伴随矩阵}。
\end{definition}
\begin{remark}
    处理遇到$\boldsymbol{A}^\ast$的题目时常用${(\boldsymbol{A}^\ast)}^{-1}=\frac{1}{|\boldsymbol{A}|}\boldsymbol{A}$
\end{remark}

\begin{theorem}[矩阵可逆的条件]
    矩阵$\boldsymbol{A}$可逆的充要条件是$|\boldsymbol{A}|\neq 0$,且$\boldsymbol{A}^{-1}=\frac{1}{|\boldsymbol{A}|}\boldsymbol{A}^\ast$.
\end{theorem}

\begin{theorem}
    矩阵$\boldsymbol{A}$可逆的充要条件是$\boldsymbol{A}$为满秩矩阵。
\end{theorem}

\begin{theorem}
    设$\boldsymbol{A}$、$\boldsymbol{B}$都是$n$阶方阵,若$\boldsymbol{A}\boldsymbol{B}=\boldsymbol{E}$,则$\boldsymbol{B}\boldsymbol{A}=\boldsymbol{E}$,且$\boldsymbol{A}^{-1}=\boldsymbol{B},\boldsymbol{B}^{-1}=\boldsymbol{A}$
\end{theorem}
\begin{remark}
    判断矩阵$\boldsymbol{B}$是否是矩阵$\boldsymbol{A}$的逆矩阵,只需要验证$\boldsymbol{A}\boldsymbol{B}=\boldsymbol{E}$和$\boldsymbol{B}\boldsymbol{A}=\boldsymbol{E}$中一个等式成立即可。
\end{remark}

\begin{theorem}[矩阵的分解]
    设$\boldsymbol{A}$为$m\times n$矩阵,$\mathrm{r}(\boldsymbol{A})=r$,则存在$m$阶可逆矩阵$\boldsymbol{P}$和$n$阶可逆矩阵$\boldsymbol{Q}$,
    使得$\boldsymbol{A}=\boldsymbol{P}\Lambda \boldsymbol{Q}$,其中
    $$\Lambda=\left( \begin{array}{ll}
        \boldsymbol{E}_r & \boldsymbol{O} \\
        \boldsymbol{O} & \boldsymbol{O}_{(m-r)\times(n-r)}
    \end{array}\right)$$
\end{theorem}

\begin{theorem}
    任一$n$阶可逆矩阵$\boldsymbol{A}$均可以表示成有限个$n$阶初等矩阵的乘积。进一步,任一可逆矩阵可以只经过行的初等变换化为单位阵,
    也可以只经过列的初等变换化为单位阵。
\end{theorem}

\begin{remark}
    该定理告诉我们一个求逆矩阵的方法:将$n\times 2n$矩阵$(\boldsymbol{A}\quad\boldsymbol{E})$经过一系列行的初等变换化为$n\times 2n$矩阵$(\boldsymbol{E}\quad\boldsymbol{B})$,则
    $\boldsymbol{B}=\boldsymbol{A}^{-1}$。

    也可以用列的初等变换来求逆矩阵,即
    $$\left(\begin{array}{c}
        \boldsymbol{A}\\
        \boldsymbol{E}        
    \end{array}\right) \underrightarrow {\text{仅用初等列变换}}\left(\begin{array}{c}
        \boldsymbol{E}\\
        \boldsymbol{A}^{-1}        
    \end{array}\right)$$

    这种求逆的方法也可用于一类矩阵方程的求解。设$\boldsymbol{A},\boldsymbol{B}$均为$n$阶方阵,且$\boldsymbol{A}$可逆,则矩阵方程
    $\boldsymbol{A}\boldsymbol{X}=\boldsymbol{B}$有解$\boldsymbol{X}=\boldsymbol{A}^{-1}\boldsymbol{B}$,可通过只做初等行变换
    将$$\left(\begin{array}{cc}
        \boldsymbol{A} & \boldsymbol{B}\\
    \end{array}\right) \longrightarrow  \left(\begin{array}{cc}
        \boldsymbol{E} & \boldsymbol{A}^{-1}\boldsymbol{B}\\
    \end{array}\right)$$来得到方程的具体解$\boldsymbol{X}=\boldsymbol{A}^{-1}\boldsymbol{B}$

    类似地,对矩阵方程 $\boldsymbol{X}\boldsymbol{A}=\boldsymbol{B}$的解$\boldsymbol{X}=\boldsymbol{B}\boldsymbol{A}^{-1}$,可通过只做初等列变换
    $$\left(\begin{array}{c}
        \boldsymbol{A} \\
        \boldsymbol{B}
    \end{array}\right) \longrightarrow  \left(\begin{array}{c}
        \boldsymbol{E} \\
        \boldsymbol{B}\boldsymbol{A}^{-1}
    \end{array}\right)$$来得到。
\end{remark}

\begin{theorem}
    设$\boldsymbol{A}$是$m\times n$矩阵,$\boldsymbol{P},\boldsymbol{Q}$分别是$m$和$n$阶可逆矩阵,则
    $$\mathrm{r}(\boldsymbol{A})=\mathrm{r}(\boldsymbol{P}\boldsymbol{A})=\mathrm{r}(\boldsymbol{A}\boldsymbol{Q})=\mathrm{r}(\boldsymbol{P}\boldsymbol{A}\boldsymbol{Q})$$
\end{theorem}

\subsection{向量组的线性相关与线性无关}
\begin{definition}[向量的线性组合,线性表示]
    给定$n$维向量组$A:\boldsymbol{\alpha}_1,\boldsymbol{\alpha}_2,\cdots,\boldsymbol{\alpha}_m$和同维向量$\boldsymbol{\beta}$,如果存在一组数$k_1,k_2,\cdots,k_m$,使
    $$\boldsymbol{\beta}=k_1\boldsymbol{\alpha}_1+k_2\boldsymbol{\alpha}_2+\cdots+k_m\boldsymbol{\alpha}_m$$
    则称向量$\boldsymbol{\beta}$是向量组$A$的一个线性组合或称向量$\boldsymbol{\beta}$可由向量组$A$线性表示。
\end{definition}

\begin{definition}[等价向量组]
    设有两个向量组$A:\boldsymbol{\alpha}_1,\boldsymbol{\alpha}_2,\cdots,\boldsymbol{\alpha}_m$及$B:\boldsymbol{\beta}_1,\boldsymbol{\beta}_2,\cdots,\boldsymbol{\beta}_l$,若$A$组中的每一个向量都可以
    由向量组$B$线性表示,则称向量组$A$可由向量组$B$线性表示。若两个向量组$A,B$可以相互线性表示,则称这两个{\heiti 向量组等价}。
\end{definition}

\begin{definition}[线性相关与线性无关]
    给一向量组$A:\boldsymbol{\alpha}_1,\boldsymbol{\alpha}_2,\cdots,\boldsymbol{\alpha}_m$,如果存在不全为零的数$k_1,k_2,\cdots,k_m$,使
    $$k_1\boldsymbol{\alpha}_1+k_2\boldsymbol{\alpha}_2+\cdots+k_m\boldsymbol{\alpha}_m=\boldsymbol{\theta}$$
    则称向量组$A$是{\heiti 线性相关}的。如果只有当$k_1=k_2=\cdots=k_m=0$时,上述等式才成立,则称这组向量是{\heiti 线性无关}的。
\end{definition}

\begin{theorem}
    关于线性相关与线性无关的基本结论:
    \begin{enumerate}[(1)]
        \item 一个向量线性相关的充要条件是$\boldsymbol{\alpha}=\boldsymbol{\theta}$;
        \item 包含零向量的向量组必是线性相关的;
        \item 如果一向量组的部分向量组线性相关,则该向量组也线性相关;
        \item 如果一个向量组线性无关,则其中任一个部分向量组也线性无关.
    \end{enumerate}
\end{theorem}

\begin{theorem}
    向量组$A:\boldsymbol{\alpha}_1,\boldsymbol{\alpha}_2,\cdots,\boldsymbol{\alpha}_m(m\geq 2)$线性相关的充要条件是,向量组$A$中至少有一个向量可由其余
    $m-1$个向量线性表示。
\end{theorem}

\begin{theorem}
    设向量组$A:\boldsymbol{\alpha}_1,\boldsymbol{\alpha}_2,\cdots,\boldsymbol{\alpha}_r$线性无关,而向量组$B:\boldsymbol{\alpha}_1,\boldsymbol{\alpha}_2,\cdots,\boldsymbol{\alpha}_r,\boldsymbol{\beta}$线性相关
    ,则向量$\boldsymbol{\beta}$必可由向量组$A$线性表示,并且表示式是唯一的。
\end{theorem}

\begin{theorem}
    $n$维列向量组$A:\boldsymbol{\alpha}_1,\boldsymbol{\alpha}_2,\cdots,\boldsymbol{\alpha}_r$线性相关的充要条件是$\mathrm{r}(\boldsymbol{A})<r$,其中矩阵
    $\boldsymbol{A}=\left(\begin{array}{cccc}
        \boldsymbol{\alpha}_1 & \boldsymbol{\alpha}_2 & \cdots & \boldsymbol{\alpha}_r 
    \end{array}\right)$。换言之,该向量组$A:\boldsymbol{\alpha}_1,\boldsymbol{\alpha}_2,\cdots,\boldsymbol{\alpha}_r$线性无关的充要条件是$\mathrm{r}(\boldsymbol{A})=r$
\end{theorem}

\begin{theorem}
    向量的个数$m$大于其维数$n$,则向量组线性相关。
\end{theorem}

\begin{theorem}
    $n$个$n$维向量线性无关的充要条件是其行列式不为零。
\end{theorem}

\begin{theorem}
    设$\boldsymbol{A}=(a_{ij})_{n\times m}$的秩$\mathrm{r}(\boldsymbol{A})=r\leq m$,且$\boldsymbol{A}$的某$r$列(行)
    所组成的矩阵含有不等于零的$r$阶子式,则此$r$个列(行)向量线性无关。
\end{theorem}

\begin{definition}[极大无关组]
    设$\boldsymbol{\alpha}_1,\boldsymbol{\alpha}_2,\cdots,\boldsymbol{\alpha}_r$是某一向量组的部分组,满足
    \begin{enumerate}[(1)]
        \item $\boldsymbol{\alpha}_1,\boldsymbol{\alpha}_2,\cdots,\boldsymbol{\alpha}_r$线性无关;
        \item 在原向量组中任取向量$\boldsymbol{\alpha}$,向量组$\boldsymbol{\alpha}_1,\boldsymbol{\alpha}_2,\cdots,\boldsymbol{\alpha}_r,\boldsymbol{\alpha}$都线性相关,则称向量组
        $\boldsymbol{\alpha}_1,\boldsymbol{\alpha}_2,\cdots,\boldsymbol{\alpha}_r$是原向量组的一个极大线性无关组,简称{\heiti 极大无关组}。
    \end{enumerate}
\end{definition}

\begin{theorem}
    设$\boldsymbol{\alpha}_{i_1},\boldsymbol{\alpha}_{i_2},\cdots,\boldsymbol{\alpha}_{i_r}$是向量组$\boldsymbol{\alpha}_1,\boldsymbol{\alpha}_2,\cdots,\boldsymbol{\alpha}_m$的一个极大线性无关组,则
    向量组$\boldsymbol{\alpha}_1,\boldsymbol{\alpha}_2,\cdots,\boldsymbol{\alpha}_m$中任一向量均可由$\boldsymbol{\alpha}_{i_1},\boldsymbol{\alpha}_{i_2},\cdots,\boldsymbol{\alpha}_{i_r}$线性表示,且表示法唯一。
\end{theorem}

\begin{theorem}
    向量组与它的任意一个极大线性无关组等价。
\end{theorem}

\begin{theorem}
    一个向量组的各个极大无关组之间是等价的。
\end{theorem}

\begin{theorem}
    两个向量组等价的充要条件是一组的一个极大无关组与另一组的一个极大无关组等价。
\end{theorem}

\begin{theorem}
    一个向量组的各个极大无关组所含向量的个数相同。
\end{theorem}

\begin{definition}[向量组的秩]
    向量组$A:\boldsymbol{\alpha}_1,\boldsymbol{\alpha}_2,\cdots,\boldsymbol{\alpha}_m$的一个极大无关组所含向量的个数定义为该{\heiti 向量组的秩},记为
    $\mathrm{r}\{\boldsymbol{\alpha}_1,\boldsymbol{\alpha}_2,\cdots,\boldsymbol{\alpha}_m\}$。规定仅含零向量的向量组的秩为零。
\end{definition}

\begin{definition}
    设$A:\boldsymbol{\alpha}_1,\boldsymbol{\alpha}_2,\cdots,\boldsymbol{\alpha}_m,B:\boldsymbol{\beta}_1,\boldsymbol{\beta}_2,\cdots,\boldsymbol{\beta}_n$是两个同维数的向量组,若向量组$A$可以由向量组
    $B$线性表示,则必有$\mathrm{r}\{\boldsymbol{\alpha}_1,\boldsymbol{\alpha}_2,\cdots,\boldsymbol{\alpha}_m\}\leq \mathrm{r}\{\boldsymbol{\beta}_1,\boldsymbol{\beta}_2,\cdots,\boldsymbol{\beta}_n\}$。
    进一步有:等价的向量组必有相同的秩。
\end{definition}

\begin{definition}
    矩阵$\boldsymbol{A}={(a_{ij})}_{m\times n}$的行向量组$\boldsymbol{\alpha}_1,\boldsymbol{\alpha}_2,\cdots,\boldsymbol{\alpha}_m$的秩称为矩阵$\boldsymbol{A}$的{\heiti 行秩};
    $\boldsymbol{A}$的列向量组$\boldsymbol{\beta}_1,\boldsymbol{\beta}_2,\cdots,\boldsymbol{\beta}_n$的秩称为矩阵$\boldsymbol{A}$的{\heiti 列秩}.
\end{definition}

\begin{theorem}
    任一矩阵的秩和其行秩、列秩都相等。
\end{theorem}

\begin{theorem}
    关于矩阵和、矩阵乘积的秩的几个重要结果:
    \begin{enumerate}[(1)]
        \item $\mathrm{r}(\boldsymbol{A}+\boldsymbol{B})\leq \mathrm{r}(\boldsymbol{A})+\mathrm{r}(\boldsymbol{B})$;
        \item $\mathrm{r}(\boldsymbol{A}\boldsymbol{B})\leq \min\{\mathrm{r}(\boldsymbol{A}),\mathrm{r}(\boldsymbol{B})\}$;
        \item 若$\boldsymbol{A},\boldsymbol{B}$均为$n$阶方阵,则$\mathrm{r}(\boldsymbol{A}\boldsymbol{B})\geq \mathrm{r}(\boldsymbol{A})+\mathrm{r}(\boldsymbol{B})-n$
    \end{enumerate}
\end{theorem}