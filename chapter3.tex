\section{线性方程组解的结构}
\subsection{高斯消元法与矩阵的行变换}

\begin{definition}
    一般的线性方程组表示为
    \begin{equation}\label{3.1}\tag{3.1}
    \left\{\begin{array}{c}
    a_{11}x_{1}+a_{12}x_{2}+\cdots+a_{1n}x_n=b_1\\
    a_{21}x_{1}+a_{22}x_{2}+\cdots+a_{2n}x_n=b_2\\
    \cdots\cdots\\
    a_{m1}x_{1}+a_{m2}x_{2}+\cdots+a_{mn}x_n=b_m\\
    \end{array}\right.\end{equation}
    由第二章矩阵和向量的知识,线性方程组\eqref{3.1}也可表示成
    \begin{equation}\label{3.2}\tag{3.2}
        x_1\boldsymbol{\alpha_1}+x_2\boldsymbol{\alpha_2}+\cdots+x_n\boldsymbol{\alpha_n}=\boldsymbol{b}
    \end{equation}
    其中向量
    $$\boldsymbol{\alpha_1}=\left(\begin{array}{c}
        a_{11}\\
        a_{21}\\
        \vdots\\
        a_{m1}
    \end{array}\right),\quad
    \boldsymbol{\alpha_2}=\left(\begin{array}{c}
        a_{12}\\
        a_{22}\\
        \vdots\\
        a_{m2}
    \end{array}\right),\quad
    \cdots,\quad
    \boldsymbol{\alpha_n}=\left(\begin{array}{c}
        a_{1n}\\
        a_{2n}\\
        \vdots\\
        a_{mn}
    \end{array}\right),\quad
    \boldsymbol{b}=\left(\begin{array}{c}
        b_{1}\\
        b_{2}\\
        \vdots\\
        b_{m}
    \end{array}\right),$$
    求解方程组\eqref{3.1}即为求解式\eqref{3.2},也就是求$\boldsymbol{b}$表示为$\boldsymbol{\alpha_1},\cdots,\boldsymbol{\alpha_n}$的线性组合。
    线性方程组\eqref{3.1}还可以表示为矩阵的形式
    \begin{equation}\label{3.3}\tag{3.3}
        \boldsymbol{A}\boldsymbol{x}=\boldsymbol{b}
    \end{equation}
    其中$$
    \boldsymbol{A}=\left(\begin{array}{cccc}
      a_{11} & a_{12} & \cdots & a_{1n}\\  
      a_{21} & a_{22} & \cdots & a_{2n}\\ 
      \vdots & \vdots &  & \vdots\\ 
      a_{m1} & a_{m2} & \cdots & a_{mn}\\ 
    \end{array}\right),\quad
    \boldsymbol{x}=\left(\begin{array}{c}
        x_1\\
        x_2\\
        \vdots\\
        x_n
    \end{array}\right),\quad
    \boldsymbol{b}=\left(\begin{array}{c}
        b_1\\
        b_2\\
        \vdots\\
        b_m
    \end{array}\right),
    $$
    称$\boldsymbol{A}$为方程组\eqref{3.1}的{\heiti 系数矩阵},$\boldsymbol{x}$为{\heiti 未知向量},$\boldsymbol{b}$为{\heiti 右端向量}。
    使方程组\eqref{3.3}成立的已知向量称为该方程组的{\heiti 解向量}。将系数矩阵与右端向量合在一起构成的矩阵$\boldsymbol{B}=(\boldsymbol{A} \quad \boldsymbol{b})$,
    称为该方程组的{\heiti 增广矩阵},也可用增广矩阵来表示线性方程组。
\end{definition}

\begin{theorem}[高斯消元法]
    通过一系列的行初等变换将增广矩阵$\boldsymbol{B}$变换成行简化梯形矩阵时,即可得到方程组的解。
\end{theorem}

\subsection{线性方程组的可解性}
\begin{theorem}
    线性方程组\eqref{3.1}有解的充要条件是系数矩阵的秩等于增广矩阵的秩.且当$\mathrm{r}(\boldsymbol{A})=\mathrm{r}(\boldsymbol{B})=n$
    时,方程组有唯一解;而当$\mathrm{r}(\boldsymbol{A})=\mathrm{r}(\boldsymbol{B})<n$时,方程组有无穷多组解。
\end{theorem}

\subsection{线性方程组解的性质与结构}
\begin{theorem}
    若$\boldsymbol{\alpha_1},\boldsymbol{\alpha_2}$是方程组$\boldsymbol{A}\boldsymbol{x}=\boldsymbol{\theta}$的解,则其线性组合
    $k_1\boldsymbol{\alpha_1}+k_2\boldsymbol{\alpha_2}$也是该方程组的解。
\end{theorem}

\begin{definition}[齐次线性方程组、基础解系]
    右端为零的线性方程组称为{\heiti 齐次线性方程组};能线性表示出齐次方程组所有解的极大无关向量组称为该齐次线性方程组的{\heiti 基础解系}。
\end{definition}

\begin{definition}[方程组的特解、通解]
    方程组的某一个解称为方程组的{\heiti 特解},方程组所有解的集合称为方程组的{\heiti 通解}。    
\end{definition}

\begin{theorem}
    设$\boldsymbol{A}\in \mathbf{R}^{m\times n}$,若$\mathrm{r}(\boldsymbol{A})=n$,则$\boldsymbol{A}\boldsymbol{x}=\boldsymbol{\theta}$只有零解;
    若$\mathrm{r}(\boldsymbol{A})<n$,则$\boldsymbol{A}\boldsymbol{x}=\boldsymbol{\theta}$有非零解。
\end{theorem}
\begin{remark}
    方程组$\boldsymbol{A}\boldsymbol{x}=\boldsymbol{\theta}$,$\boldsymbol{A}\in \mathbf{R}^{m\times n}$有非零解的充要条件是$|\boldsymbol{A}|=0$
\end{remark}
\begin{remark}
    若$\boldsymbol{A}\in \mathbf{R}^{m\times n}$,且$m<n$,则$\boldsymbol{A}\boldsymbol{x}=\boldsymbol{\theta}$有非零解。
\end{remark}

\begin{theorem}
    $\boldsymbol{A}$经过适当的行初等变换可以化为如下形式的行简化梯形矩阵
    $$\begin{array}{c}\left(\begin{array}{cccccccccc}
        1 & d_{12} & \cdots & 0 & d_{1i_2+1} & \cdots & 0 & d_{1i_r+1} & \cdots & d_{1n} \\
        0 & 0 & \cdots & 1 & d_{2i_2+1} & \cdots & 0 & d_{21i_r+1} & \cdots & d_{2n} \\
        \vdots & \vdots & & \vdots & \vdots &  & \vdots & \vdots &  & \vdots\\
        0 & 0 & \cdots & 0 & 0 & \cdots & 1 & d_{ri_r+1} & \cdots & d_{rn} \\
        0 & 0 & \cdots & 0 & 0 & \cdots & 0 & 0 & \cdots & 0 \\
        \vdots & \vdots & & \vdots & \vdots &  & \vdots & \vdots &  & \vdots\\
        0 & 0 & \cdots & 0 & 0 & \cdots & 0 & 0 & \cdots & 0 
    \end{array}\right)\\
    \begin{array}{lccccccccc}
        1 & 2{\color{white} 0} & \cdots & i_2 & i_2+1 & \cdots & i_r & i_r+1 & \cdots & {\color{white} 0} n
    \end{array}
    \end{array}$$
    其中最后一行是矩阵所在列的列标号。对$n-r$个自由变量$x_2,\cdots,x_{i_2-1},x_{i_2+1},\cdots, x_{i_r-1}, x_{i_r+1},\cdots, x_n $分别取$n-r$
    组数据$(1,0,\cdots,0),(0,1,0,\cdots,0),\cdots,(0,\cdots,0,1)$,则可得$n-r$组非自由变量$x_1,x_{i_2},\cdots,x_{i_r}$的值,从而构成$n-r$组方程组
    的解,设该$n-r$组解的解向量为$\boldsymbol{\alpha_1},\boldsymbol{\alpha_2},\cdots\boldsymbol{\alpha_{n-r}}$,则它们就是方程组的一个基础解系。
    方程组的通解为:
    $$k_1\boldsymbol{\alpha_1}+\cdots+k_{n-r}\boldsymbol{\alpha_{n-r}},\quad \mbox{其中}k_1,\cdots,k_{n-r}\in \mathbf{R}\mbox{为任意常数}$$
\end{theorem}
\begin{remark}
    若系数矩阵简化后的行简化梯形矩阵为
    $$\left(\begin{array}{ccccccc}
        1 & 0 & \cdots & 0 & d_{1,r+1} & \cdots & d_{1n} \\
        0 & 1 & \cdots & 0 & d_{2,r+1} & \cdots & d_{2n} \\
        \vdots & \vdots &  & \vdots & \vdots &  & \vdots \\
        0 & 0 & \cdots & 1 & d_{r,r+1} & \cdots & d_{rn} \\
        0 & 0 & \cdots & 0 & 0 & \cdots & 0 \\
        \vdots & \vdots &  & \vdots & \vdots &  & \vdots \\
        0 & 0 & \cdots & 0 & 0 & \cdots & 0 \\
    \end{array}\right)$$
    则
    $$\boldsymbol{\alpha_1}=\left(\begin{array}{c}
        -d_{1,r+1}\\
        \vdots\\
        -d_{r,r+1}\\
        1\\
        0\\
        \vdots\\
        0
    \end{array}\right),\quad
    \boldsymbol{\alpha_2}=\left(\begin{array}{c}
        -d_{1,r+2}\\
        \vdots\\
        -d_{r,r+2}\\
        0\\
        1\\
        \vdots\\
        0
    \end{array}\right),\quad
    \cdots,\quad
    \boldsymbol{\alpha_{n-r}}=\left(\begin{array}{c}
        -d_{1n}\\
        \vdots\\
        -d_{rn}\\
        0\\
        \vdots\\
        0\\
        1
    \end{array}\right),$$
    是齐次方程组的一个基础解系。
\end{remark}

\begin{theorem}
    若$\boldsymbol{A}\in \mathbf{R}^{m\times n}$,则$\mathrm{r}(\boldsymbol{A})+\mathrm{r}(N(\boldsymbol{A}))=n$,
    其中$N(\boldsymbol{A})$表示$\boldsymbol{A}\boldsymbol{x}=\boldsymbol{\theta}$的基础解系为列构成的矩阵。
\end{theorem}

\begin{theorem}
    若$\boldsymbol{A}\boldsymbol{\eta}=\boldsymbol{b}(\boldsymbol{b}\neq \boldsymbol{\theta})$,则$\boldsymbol{A}\boldsymbol{x}=\boldsymbol{b}$
    的通解可以表示为
    $$\boldsymbol{\eta}+\boldsymbol{\alpha}$$
    其中$\boldsymbol{\alpha}$为$\boldsymbol{A}\boldsymbol{x}=\boldsymbol{\theta}$的解。若$\boldsymbol{A}\boldsymbol{x}=\boldsymbol{\theta}$的基础解系为$\boldsymbol{\alpha_1},\cdots,\boldsymbol{\alpha_r}$
    ,则$\boldsymbol{A}\boldsymbol{x}=\boldsymbol{b}$的通解为:
    $\boldsymbol{\eta}+k_1\boldsymbol{\alpha_1}+\cdots+k_r\boldsymbol{\alpha_r}$,
    其中$k_1,\cdots,k_r\in \mathbf{R}$为任意实数。
\end{theorem}
