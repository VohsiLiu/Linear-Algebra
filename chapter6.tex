\section{线性空间与线性变换}
\subsection{线性空间的定义}
    \begin{definition}[数环]
        设$\mathfrak{R}$是非空数集,其中任何两个数之和、差与积仍属于$\mathfrak{R}$(即$\mathfrak{R}$关于加、减、乘法运算是封闭的),则称$\mathfrak{R}$是一个{\heiti 数环}。
    \end{definition}

    \begin{definition}[数域]
        若$K$是至少含有两个互异数的数环,且其中任何两数$a$与$b$之商$(b\neq 0)$仍属于$K$,则称$K$是一个{\heiti 数域}。
    \end{definition}

    \begin{definition}[线性空间]
        设$V$是一个非空集合,$K$是一个数域,$V$满足以下两个条件:
        \begin{enumerate}[(1)]
            \item 在$V$中定义一个封闭的加法运算,即当$x,y\in V$时,有唯一的$z=x+y\in V$,且此加法运算满足下面4条性质:
            \begin{enumerate}[1)]
                \item $x+y=y+x$(交换律)
                \item $x+(y+z)=(x+y)+z$(结合律)
                \item 存在
            \end{enumerate}
        \end{enumerate}
    \end{definition}


\subsection{线性空间的基、维数与坐标}



\subsection{线性空间的子空间}




\subsection{线性变换}



\subsection{线性变换的特征值和特征向量}