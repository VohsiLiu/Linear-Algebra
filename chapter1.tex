\section{行列式}

\subsection{二阶与三阶行列式}

\begin{definition}[二阶行列式]
  将4个可以进行乘法与加法运算的元素$a,b,c,d$排成两行两列,引用记号
  $$\left|\begin{array}{ll}
    a & b \\
    c & d
    \end{array}\right|=ad-bc$$
  并称之为{\heiti 二阶行列式}。行列式也可简记为$\Delta$、$D$等
\end{definition}

\begin{theorem}
  对于方程组
  \begin{equation}\label{1}
  \left\{\begin{array}{l}a_{11} x_{1}+a_{12} x_{2}=b_{1} \\ a_{21} x_{1}+a_{22} x_{2}=b_{2}\end{array}\right.
  \end{equation}
  记$$\Delta= \left|\begin{array}{ll}
    a_{11} & a_{12} \\
    a_{21} & a_{22}
    \end{array}\right|=a_{11}a_{22}-a_{12}a_{21}$$
    $$\Delta_1= \left|\begin{array}{ll}
      b_{1} & a_{12} \\e
      b_{22} & a_{22}
      \end{array}\right|=b_{1}a_{22}-a_{12}b_{2}$$
    $$\Delta_2= \left|\begin{array}{ll}
      a_{11} & b_{1} \\
      a_{21} & b_{2}
      \end{array}\right|=a_{11}b_{2}-b_{1}a_{21}$$
      有如下结论:
    \begin{enumerate}
      \item 若$\Delta \neq 0$,则方程组\eqref{1} 有唯一解:$\displaystyle{x_1=\frac{\Delta_1}{\Delta},x_2=\frac{\Delta_2}{\Delta}}$
      \item 若$\Delta=0$,但$\Delta_1,\Delta_2$不全为零则方程组\eqref{1} 无解
      \item 若$\Delta=\Delta_1=\Delta_2=0$,则方程组\eqref{1}有无穷多组解
    \end{enumerate}
\end{theorem}

\begin{definition}[三阶行列式]
  设有9个可以进行乘法和加法运算的元素排成三行三列,引用记号
  $$\left|\begin{array}{lll}
    a_{11} & a_{12} & a_{13} \\
    a_{21} & a_{22} & a_{23} \\
    a_{31} & a_{32} & a_{33}
    \end{array}\right|=a_{11} a_{22} a_{33}+a_{12} a_{23} a_{31}+a_{13} a_{21} a_{32}-a_{13} a_{22} a_{31}-a_{12} a_{21} a_{33}-a_{11} a_{23} a_{32}$$
    并称之为{\heiti 三阶行列式},其中$a_{ij}(i,j=1,2,3)$为该行列式的元素。
\end{definition}

\begin{theorem}
  对于方程组
  \begin{equation}\label{2}
    \left\{\begin{array}{l}
      a_{11} x_{1}+a_{12} x_{2}+a_{13} x_{3}=b_{1} \\
      a_{21} x_{1}+a_{22} x_{2}+a_{23} x_{3}=b_{2} \\
      a_{31} x_{1}+a_{32} x_{2}+a_{33} x_{3}=b_{3}
      \end{array}\right.
  \end{equation}
  记$$\Delta_{1}=\left|\begin{array}{ccc}
    b_{1} & a_{12} & a_{13} \\
    b_{2} & a_{22} & a_{23} \\
    b_{3} & a_{32} & a_{33}
    \end{array}\right|, \quad \Delta_{2}=\left|\begin{array}{ccc}
    a_{11} & b_{1} & a_{13} \\
    a_{21} & b_{2} & a_{23} \\
    a_{31} & b_{3} & a_{33}
    \end{array}\right|, \quad \Delta_{3}=\left|\begin{array}{ccc}
    a_{11} & a_{12} & b_{1} \\
    a_{21} & a_{22} & b_{2} \\
    a_{31} & a_{32} & b_{3}
    \end{array}\right|$$
    有如下结论:
    \begin{enumerate}
      \item 若$\Delta\neq 0$,则方程组\eqref{2}有唯一解$\displaystyle{x_1=\frac{\Delta_1}{\Delta},x_2=\frac{\Delta_2}{\Delta},x_3=\frac{\Delta_3}{\Delta}}$
      \item 若$\Delta=0$,而$\Delta_1,\Delta_2,\Delta_3$不全为0,则方程组\eqref{2}无解
      \item 若$\Delta=\Delta_1=\Delta_2=\Delta_3=0$,则方程组\eqref{2}可能无解也可能有无穷多组解
    \end{enumerate}
\end{theorem}

\subsection{$n$阶行列式}
\begin{definition}[$n$阶行列式]
  设有$n^2$个可以进行加法和乘法运算的元素排成$n$行$n$列,引用记号
  $$
  D_{n}=\left|\begin{array}{cccc}
  a_{11} & a_{12} & \cdots & a_{1 n} \\
  a_{21} & a_{22} & \cdots & a_{2 n} \\
  \vdots & \vdots & & \vdots \\
  a_{n 1} & a_{n 2} & \cdots & a_{n n}
  \end{array}\right|
  $$
  称它为{\heiti $n$阶行列式},它是一个算式,有时也用记号$|a_{ij}|_{n\times n}$表示。

  其数值可归纳定义为:当$n=1$时,一阶行列式的值定义为$D_1=\det (a_{11})=a_{11}$;当$n\geq 2$时,
  $$D_n=a_{11}A_{11}+a_{12}A_{12}+\cdots+a_{1n}A_{1n}=\sum _{j=1}^n a_{1j}A_{1j}$$
  其中$$A_{ij}={(-1)}^{i+j}M_{ij}$$
  而$$
  M_{i j}=\left|\begin{array}{cccccc}
  a_{11} & \cdots & a_{1, j-1} & a_{1, j+1} & \cdots & a_{1 n} \\
  \vdots & & \vdots & \vdots & & \vdots \\
  a_{i-1,1} & \cdots & a_{i-1, j-1} & a_{i-1, j+1} & \cdots & a_{i-1, n} \\
  a_{i+1,1} & \cdots & a_{i+1, j-1} & a_{i+1, j+1} & \cdots & a_{i+1, n} \\
  \vdots & & \vdots & \vdots & & \vdots \\
  a_{n 1} & \cdots & a_{n, j-1} & a_{n, j+1} & \cdots & a_{n n}
  \end{array}\right|
  $$
  并称$M_{ij}$为元素$a_{ij}$的{\heiti 余子式},$A_{ij}$为元素$a_{ij}$的{\heiti 代数余子式}。显然$M_{ij}$
  为一个$n-1$阶的行列式,它是在$D_n$中划去元素$a_{ij}$所在的第$i$行和第$j$列后得到的一个行列式。
\end{definition}

\begin{remark}
  三角行列式的值为主对角线上元素的乘积。
\end{remark}

\begin{definition}[转置行列式]
  设$$
  A=\left|\begin{array}{cccc}
  a_{11} & a_{12} & \cdots & a_{1 n} \\
  a_{21} & a_{22} & \cdots & a_{2 n} \\
  \vdots & \vdots & & \vdots \\
  a_{n 1} & a_{n 2} & \cdots & a_{n n}
  \end{array}\right|, \quad A^{\prime}=\left|\begin{array}{cccc}
  a_{11} & a_{21} & \cdots & a_{n 1} \\
  a_{12} & a_{22} & \cdots & a_{n 2} \\
  \vdots & \vdots & & \vdots \\
  a_{1 n} & a_{2 n} & \cdots & a_{n n}
  \end{array}\right|
  $$
  称$A^{\prime}$为行列式$A$的{\heiti 转置行列式}(也可以表示为$A^{\mathrm{T}}$)。
  显然$A^{\prime}$是行列式$A$的行和列互换之后得到的行列式。
\end{definition}

\begin{theorem}
  $n$阶行列式的性质:
  \begin{itemize}
    \item 行列式与它的转置行列式的值相等。
    \item 对调两行(列)的位置,行列式的值相差一个负号
    \item 两行(列)相等的行列式的值为0
    \item 行列式可以按任意一行(列)展开
    \item 行列式的任一行(列)元素的公因子可以提到行列式外面
    \item 若行列式的某两行(列)成比例,则该行列式的值为零
    \item 若行列式的第$i$行(列)的每一个元素都可以表示为两数的和,则该行列式可以表示为两个行列式之和
    \item 行列式任一行(列)的元素与另一行(列)元素的代数余子式对应乘积之和为零\\即,若设$A=|a_{ij}|_{n\times n}$,则有
    $$
    \sum_{k=1}^{n} a_{i k} A_{j k}=a_{i 1} A_{j 1}+a_{i 2} A_{j 2}+\cdots+a_{i n} A_{j n}=\left\{\begin{array}{ll}
    A, & i=j \\
    0, & i \neq j
    \end{array}\right.
    $$
    或
    $$
    \sum_{k=1}^{n} a_{ki} A_{kj}=a_{1i} A_{1j}+a_{2i} A_{2j}+\cdots+a_{ni} A_{nj}=\left\{\begin{array}{ll}
    A, & i=j \\
    0, & i \neq j
    \end{array}\right.
    $$
  \end{itemize}
\end{theorem}

\begin{theorem}[克莱姆法则]
  对于$n$元线性方程组
    \begin{equation}\label{3}
      \left\{\begin{array}{l}
        a_{11} x_{1}+a_{12} x_{2}+\cdots+a_{1n} x_{3}=b_{1} \\
        a_{21} x_{1}+a_{22} x_{2}+\cdots+a_{2n} x_{3}=b_{2} \\
        \cdots\cdots\\
        a_{n1} x_{1}+a_{n2} x_{2}+\cdots+a_{nn} x_{n}=b_{n}
        \end{array}\right.
    \end{equation}
    若系数行列式
    $$
    D=\left|\begin{array}{cccc}
    a_{11} & a_{12} & \cdots & a_{1 n} \\
    a_{21} & a_{22} & \cdots & a_{2 n} \\
    \vdots & \vdots & & \vdots \\
    a_{n 1} & a_{n 2} & \cdots & a_{n n}
    \end{array}\right| \neq 0
    $$
    则方程组\eqref{3}有解,且解是唯一的,这个解可以表示为
    $$x_j=\frac{D_j}{D}\qquad(j=1,2,\cdots,n)$$
    其中$D_j(j=1,2,\cdots,n)$是将$D$的第$j$列元素$a_{1j},a_{2j},\cdots,a_{nj}$换成方程组右端的常数项
    $b_1,b_2,\cdots,b_n$所得到的行列式,即
    $$
    D_{j}=\left|\begin{array}{ccccccc}
    a_{11} & \cdots & a_{1, j-1} & b_{1} & a_{1, j+1} & \cdots & a_{1 n} \\
    a_{21} & \cdots & a_{2, j-1} & b_{2} & a_{2, j+1} & \cdots & a_{2 n} \\
    \vdots & & \vdots & \vdots & \vdots & & \vdots \\
    a_{n 1} & \cdots & a_{n, j-1} & b_{n} & a_{n, j+1} & \cdots & a_{n n}
    \end{array}\right| \quad(j=1,2, \cdots, n)
    $$

    考虑方程组\eqref{3}的一个特殊情形,方程组\eqref{3}右端的常数项$b_1,b_2,\cdots,b_n$全部为零,即
    \begin{equation}\label{4}   
    \left\{\begin{array}{l}
      a_{11} x_{1}+a_{12} x_{2}+\cdots+a_{1n} x_{3}=0 \\
      a_{21} x_{1}+a_{22} x_{2}+\cdots+a_{2n} x_{3}=0\\
      \cdots\cdots\\
      a_{n1} x_{1}+a_{n2} x_{2}+\cdots+a_{nn} x_{n}=0
      \end{array}\right.
    \end{equation}
      这样的方程组称为{\heiti 齐次线性方程组},而方程组\eqref{3}称为{\heiti 非齐次线性方程组}。
\end{theorem}

\begin{theorem}
  含有$n$个未知数$n$个方程的齐次线性方程组\eqref{4}若有非零解,则它的系数行列式等于零。
\end{theorem}
