\section{矩阵的特征值与特征向量}

\subsection{相似矩阵}
\begin{definition}[相似矩阵]
    对于同阶方阵$\boldsymbol{A}$与$\boldsymbol{B}$,如果存在可逆矩阵$\boldsymbol{P}$,使得
    $$\boldsymbol{B}=\boldsymbol{P}^{-1}\boldsymbol{AP}$$
    则称$\boldsymbol{A}${\heiti 相似}于$\boldsymbol{B}$,记为$\boldsymbol{A}\sim \boldsymbol{B}$。称
    $\boldsymbol{B}$为$\boldsymbol{A}$的{\heiti 相似矩阵},而称$\boldsymbol{P}$为$\boldsymbol{A}$到$\boldsymbol{B}$的
    {\heiti 相似变换矩阵}。
\end{definition}

\begin{theorem}
    相似矩阵的性质:
    \begin{enumerate}[(1)]
        \item 若$\boldsymbol{A} \sim \boldsymbol{B}$,则$|\boldsymbol{A}|=|\boldsymbol{B}|$,从而$\boldsymbol{A}$与$\boldsymbol{B}$可逆性相同;
        \item 若$\boldsymbol{A} \sim \boldsymbol{B}$,且$\boldsymbol{A}$或$\boldsymbol{B}$可逆,则$\boldsymbol{A}^{-1} \sim \boldsymbol{B}^{-1}$;
        \item 若$\boldsymbol{A} \sim \boldsymbol{B}$,则$\boldsymbol{A}^n \sim \boldsymbol{B}^n,k\boldsymbol{A} \sim k\boldsymbol{B}$,其中$n$为自然数,$k$为任意实数;
        \item 若$\boldsymbol{A} \sim \boldsymbol{B}$,则$f(\boldsymbol{A})\sim f(\boldsymbol{B})$,其中$f(x)=a_nx^n+a_{n-1}x^{n-1}+\cdots+a_1x+a_0$为任意多项式。
    \end{enumerate}
\end{theorem}

\subsection{特征值与特征向量}
\begin{definition}[特征值、特征向量]
    设$\boldsymbol{A}$是实数域$\mathbf{R}$或复数域$\mathbf{C}$上的一个方阵,$\lambda\in \mathbf{C}$,若
    存在非零向量$\boldsymbol{\xi }$使得$\boldsymbol{A}\boldsymbol{\xi}=\lambda\boldsymbol{\xi}$,则称
    $\lambda$为矩阵$\boldsymbol{A}$的{\heiti 特征值},$\boldsymbol{\xi}$称为$\boldsymbol{A}$的属于特征值
    $\lambda$的{\heiti 特征向量}.
\end{definition}

\begin{theorem}
    设方阵$\boldsymbol{A}$有特征值$\lambda$,$\boldsymbol{\xi_1},\boldsymbol{\xi_2}$为属于$\lambda$的特征向量,则它们的任意不等于零向量的线性组合
    $\boldsymbol{\eta}=k_1\boldsymbol{\xi_1}+k_2\boldsymbol{\xi_2}(k_1,k_2\in \mathbf{R})$仍是属于$\lambda$的特征向量。
\end{theorem}

\begin{definition}[特征多项式、特征方程、特征矩阵]
    $|\lambda\boldsymbol{E}-\boldsymbol{A}|$称为$\boldsymbol{A}$的{\heiti 特征多项式};方程$|\lambda\boldsymbol{E}-\boldsymbol{A}|=0$称为$\boldsymbol{A}$
    的{\heiti 特征方程}。方程$|\lambda\boldsymbol{E}-\boldsymbol{A}|=0$的解也称为$\boldsymbol{A}$的{\heiti 特征根}.而$\lambda\boldsymbol{E}-\boldsymbol{A}$
    称为$\boldsymbol{A}$的{\heiti 特征矩阵}。
\end{definition}


\begin{theorem}
    求矩阵$\boldsymbol{A}$的全部特征值和特征向量的计算步骤:

    (1)计算行列式$|\lambda\boldsymbol{E}-\boldsymbol{A}|$,并求出$|\lambda\boldsymbol{E}-\boldsymbol{A}|=0$的全部根,即$\boldsymbol{A}$的特征值;
    
    (2)对于每个特征值$\lambda_i$,求齐次线性方程组$$(\lambda_i\boldsymbol{E}-\boldsymbol{A})\boldsymbol{x}=\boldsymbol{\theta}$$的一个基础解系$\boldsymbol{\alpha_1},\boldsymbol{\alpha_2},\cdots,\boldsymbol{\alpha_{s_i}}$;

    (3)写出$\boldsymbol{A}$属于$\lambda_i$的全部特征向量为$$k_1\boldsymbol{\alpha_1}+k_2\boldsymbol{\alpha_2}+\cdots+k_{s_i}\boldsymbol{\alpha_{s_i}}$$
    其中$k_1,k_2,\cdots,k_{s_i}$为不全为零的任意常数。而所求得的所有特征值的所有特征向量,就是$\boldsymbol{A}$的全部特征向量。
\end{theorem}

\begin{remark}
    对于重特征值$\lambda$,虽然有多个属于$\lambda$的线性无关的特征向量,但其个数不会超过$\lambda$的重数。对于单特征值,有且只有一个线性无关的特征向量。
\end{remark}

\begin{theorem}
    若$f(x)$为$x$的多项式,矩阵$\boldsymbol{A}$有特征值$\lambda$,则$f(\boldsymbol{A})$有特征值$f(\lambda)$。
\end{theorem}
\begin{remark}
    若该定理中矩阵$\boldsymbol{A}$的所有特征值为$\lambda_1,\cdots,\lambda_n$(包括相同的特征值),则$f(\boldsymbol{A})$
    的所有特征值为$f(\lambda_1),\cdots,f(\lambda_n)$.
\end{remark}
\begin{remark}
    若$n$阶可逆方阵$\boldsymbol{A}$的所有特征值为$\lambda_1,\cdots,\lambda_n$(包括相同的特征值),则$\lambda_i\neq 0,i=1,2,\cdots,n$,
    且矩阵$\boldsymbol{A}^{-1}$的所有特征值为$\lambda_1^{-1},\cdots,\lambda_n^{-1}$.
\end{remark}
\begin{remark}
    不可逆方阵$\boldsymbol{A}$必有0特征值.
\end{remark}

\begin{theorem}
    相似矩阵具有相同的特征多项式,从而它们具有相同的特征值。
\end{theorem}

\begin{definition}[迹]
    定义$$\mathrm{tr}(\boldsymbol{A})=\sum_{i=1}^n a_{ii}$$
    为矩阵$\boldsymbol{A}=(a_{ij})_{n\times n}$的迹。
\end{definition}

\begin{theorem}
    若$n$阶矩阵$\boldsymbol{A}$的特征值为$\lambda_1,\cdots,\lambda_n$,则有
    $$\mathrm{tr}(\boldsymbol{A})=\sum_{i=1}^n \lambda_i,\quad |\boldsymbol{A}|=\prod_{i=1}^n \lambda_i$$
\end{theorem}

\begin{theorem}
    相似矩阵具有相同的迹和相同的行列式.
\end{theorem}

\begin{theorem}
    设$\boldsymbol{A}$是一个快对角矩阵
    $$\boldsymbol{A}=\left(\begin{array}{cccc}
        \boldsymbol{A_1} & & & \\
         & \boldsymbol{A_2}& & \\
         & & \ddots & \\
         & & & \boldsymbol{A_m}\\
    \end{array}\right)$$
    则$\boldsymbol{A}$的特征多项式是$\boldsymbol{A_1},\boldsymbol{A_2},\cdots,\boldsymbol{A_m}$的特征多项式的乘积,
    于是$\boldsymbol{A_1},\boldsymbol{A_2},\cdots,\boldsymbol{A_m}$的所有特征值就是$\boldsymbol{A}$的所有特征值。
\end{theorem}