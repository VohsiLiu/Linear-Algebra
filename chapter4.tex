\section{矩阵的特征值与特征向量}

\subsection{相似矩阵}
\begin{definition}[相似矩阵]
    对于同阶方阵$\boldsymbol{A}$与$\boldsymbol{B}$,如果存在可逆矩阵$\boldsymbol{P}$,使得
    $$\boldsymbol{B}=\boldsymbol{P}^{-1}\boldsymbol{AP}$$
    则称$\boldsymbol{A}${\heiti 相似}于$\boldsymbol{B}$,记为$\boldsymbol{A}\sim \boldsymbol{B}$。称
    $\boldsymbol{B}$为$\boldsymbol{A}$的{\heiti 相似矩阵},而称$\boldsymbol{P}$为$\boldsymbol{A}$到$\boldsymbol{B}$的
    {\heiti 相似变换矩阵}。
\end{definition}

\begin{theorem}
    相似矩阵的性质:
    \begin{enumerate}[(1)]
        \item 若$\boldsymbol{A} \sim \boldsymbol{B}$,则$|\boldsymbol{A}|=|\boldsymbol{B}|$,从而$\boldsymbol{A}$与$\boldsymbol{B}$可逆性相同;
        \item 若$\boldsymbol{A} \sim \boldsymbol{B}$,且$\boldsymbol{A}$或$\boldsymbol{B}$可逆,则$\boldsymbol{A}^{-1} \sim \boldsymbol{B}^{-1}$;
        \item 若$\boldsymbol{A} \sim \boldsymbol{B}$,则$\boldsymbol{A}^n \sim \boldsymbol{B}^n,k\boldsymbol{A} \sim k\boldsymbol{B}$,其中$n$为自然数,$k$为任意实数;
        \item 若$\boldsymbol{A} \sim \boldsymbol{B}$,则$f(\boldsymbol{A})\sim f(\boldsymbol{B})$,其中$f(x)=a_nx^n+a_{n-1}x^{n-1}+\cdots+a_1x+a_0$为任意多项式。
    \end{enumerate}
\end{theorem}

\subsection{特征值与特征向量}
\begin{definition}[特征值、特征向量]
    设$\boldsymbol{A}$是实数域$\mathbf{R}$或复数域$\mathbf{C}$上的一个方阵,$\lambda\in \mathbf{C}$,若
    存在非零向量$\boldsymbol{\xi }$使得$\boldsymbol{A}\boldsymbol{\xi}=\lambda\boldsymbol{\xi}$,则称
    $\lambda$为矩阵$\boldsymbol{A}$的{\heiti 特征值},$\boldsymbol{\xi}$称为$\boldsymbol{A}$的属于特征值
    $\lambda$的{\heiti 特征向量}.
\end{definition}

\begin{theorem}
    设方阵$\boldsymbol{A}$有特征值$\lambda$,$\boldsymbol{\xi}_1,\boldsymbol{\xi}_2$为属于$\lambda$的特征向量,则它们的任意不等于零向量的线性组合
    $\boldsymbol{\eta}=k_1\boldsymbol{\xi}_1+k_2\boldsymbol{\xi}_2(k_1,k_2\in \mathbf{R})$仍是属于$\lambda$的特征向量。
\end{theorem}

\begin{definition}[特征多项式、特征方程、特征矩阵]
    $|\lambda\boldsymbol{E}-\boldsymbol{A}|$称为$\boldsymbol{A}$的{\heiti 特征多项式};方程$|\lambda\boldsymbol{E}-\boldsymbol{A}|=0$称为$\boldsymbol{A}$
    的{\heiti 特征方程}。方程$|\lambda\boldsymbol{E}-\boldsymbol{A}|=0$的解也称为$\boldsymbol{A}$的{\heiti 特征根}.而$\lambda\boldsymbol{E}-\boldsymbol{A}$
    称为$\boldsymbol{A}$的{\heiti 特征矩阵}。
\end{definition}


\begin{theorem}
    求矩阵$\boldsymbol{A}$的全部特征值和特征向量的计算步骤:

    (1)计算行列式$|\lambda\boldsymbol{E}-\boldsymbol{A}|$,并求出$|\lambda\boldsymbol{E}-\boldsymbol{A}|=0$的全部根,即$\boldsymbol{A}$的特征值;
    
    (2)对于每个特征值$\lambda_i$,求齐次线性方程组$$(\lambda_i\boldsymbol{E}-\boldsymbol{A})\boldsymbol{x}=\boldsymbol{\theta}$$的一个基础解系$\boldsymbol{\alpha}_1,\boldsymbol{\alpha}_2,\cdots,\boldsymbol{\alpha}_{s_i}$;

    (3)写出$\boldsymbol{A}$属于$\lambda_i$的全部特征向量为$$k_1\boldsymbol{\alpha}_1+k_2\boldsymbol{\alpha}_2+\cdots+k_{s_i}\boldsymbol{\alpha}_{s_i}$$
    其中$k_1,k_2,\cdots,k_{s_i}$为不全为零的任意常数。而所求得的所有特征值的所有特征向量,就是$\boldsymbol{A}$的全部特征向量。
\end{theorem}

\begin{remark}
    对于重特征值$\lambda$,虽然有多个属于$\lambda$的线性无关的特征向量,但其个数不会超过$\lambda$的重数。对于单特征值,有且只有一个线性无关的特征向量。
\end{remark}

\begin{theorem}
    若$f(x)$为$x$的多项式,矩阵$\boldsymbol{A}$有特征值$\lambda$,则$f(\boldsymbol{A})$有特征值$f(\lambda)$。
\end{theorem}
\begin{remark}
    若该定理中矩阵$\boldsymbol{A}$的所有特征值为$\lambda_1,\cdots,\lambda_n$(包括相同的特征值),则$f(\boldsymbol{A})$
    的所有特征值为$f(\lambda_1),\cdots,f(\lambda_n)$.
\end{remark}
\begin{remark}
    若$n$阶可逆方阵$\boldsymbol{A}$的所有特征值为$\lambda_1,\cdots,\lambda_n$(包括相同的特征值),则$\lambda_i\neq 0,i=1,2,\cdots,n$,
    且矩阵$\boldsymbol{A}^{-1}$的所有特征值为$\lambda_1^{-1},\cdots,\lambda_n^{-1}$.
\end{remark}
\begin{remark}
    不可逆方阵$\boldsymbol{A}$必有0特征值.
\end{remark}

\begin{theorem}
    相似矩阵具有相同的特征多项式,从而它们具有相同的特征值。
\end{theorem}

\begin{definition}[迹]
    定义$$\mathrm{tr}(\boldsymbol{A})=\sum_{i=1}^n a_{ii}$$
    为矩阵$\boldsymbol{A}=(a_{ij})_{n\times n}$的迹。
\end{definition}

\begin{theorem}
    若$n$阶矩阵$\boldsymbol{A}$的特征值为$\lambda_1,\cdots,\lambda_n$,则有
    $$\mathrm{tr}(\boldsymbol{A})=\sum_{i=1}^n \lambda_i,\quad |\boldsymbol{A}|=\prod_{i=1}^n \lambda_i$$
\end{theorem}

\begin{theorem}
    相似矩阵具有相同的迹和相同的行列式.
\end{theorem}

\begin{theorem}
    设$\boldsymbol{A}$是一个块对角矩阵
    $$\boldsymbol{A}=\left(\begin{array}{cccc}
        \boldsymbol{A}_1 & & & \\
         & \boldsymbol{A}_2& & \\
         & & \ddots & \\
         & & & \boldsymbol{A}_m\\
    \end{array}\right)$$
    则$\boldsymbol{A}$的特征多项式是$\boldsymbol{A}_1,\boldsymbol{A}_2,\cdots,\boldsymbol{A}_m$的特征多项式的乘积,
    于是$\boldsymbol{A}_1,\boldsymbol{A}_2,\cdots,\boldsymbol{A}_m$的所有特征值就是$\boldsymbol{A}$的所有特征值。
\end{theorem}

\subsection{矩阵可对角化的条件}
\begin{definition}[可对角化]
    若方阵$\boldsymbol{A}$相似于一个对角矩阵,则称$\boldsymbol{A}${\heiti 可对角化}。
\end{definition}

\begin{theorem}
    $n$阶矩阵可对角化的充要条件是有$n$个线性无关的特征向量;且对角矩阵的主对角线由特征值(可按任意次序)构成,
    相似变换矩阵由属于相应特征值的特征向量构成。
\end{theorem}

\begin{theorem}
    属于不同特征值的特征向量线性无关。
\end{theorem}

\begin{theorem}
    若$n$阶矩阵有$n$个互不相同的特征值,则矩阵可对角化。
\end{theorem}

\begin{theorem}
    若$\lambda_1,\lambda_2,\cdots,\lambda_m$是矩阵$\boldsymbol{A}$的不同特征值,而$\boldsymbol{A}$的属于$\lambda_i$的线性无关
    的特征向量为$\boldsymbol{\alpha}_{i1},\boldsymbol{\alpha}_{i2},\cdots,\boldsymbol{\alpha}_{is_i}(i=1,2,\cdots,m)$,则向量组
    $$\boldsymbol{\alpha}_{11},\cdots,\boldsymbol{\alpha}_{1s_1},\boldsymbol{\alpha}_{21},\cdots,\boldsymbol{\alpha}_{2s_2},\cdots,\boldsymbol{\alpha}_{m1},\cdots,\boldsymbol{\alpha}_{ms_m}$$
    线性无关。
\end{theorem}

\begin{theorem}
    设$\lambda_0$是$n$阶方阵$\boldsymbol{A}$的$k$重特征值,则$\boldsymbol{A}$的属于特征值$\lambda_0$的线性无关的特征向量个数不超过$k$.
\end{theorem}

\begin{theorem}
    $n$阶方阵$\boldsymbol{A}$可对角化的充要条件是每个$k_i$重特征值$\lambda_i$对应的特征矩阵$\lambda_i\boldsymbol{E}-\boldsymbol{A}$的秩为$n-k_i$.
\end{theorem}

\begin{theorem}
    在$\boldsymbol{A}$有重特征值时$\boldsymbol{A}$可否对角化的判定方法:

    对每个重特征值$\lambda_i$,求矩阵$\lambda_i\boldsymbol{E}-\boldsymbol{A}$的秩$r_i$,若对每个重特征值$\lambda_i$,$n-r_i$都等于
    $\lambda_i$的重数,则$\boldsymbol{A}$可以对角化;否则$\boldsymbol{A}$不可对角化。
\end{theorem}

\begin{theorem}
    矩阵$\boldsymbol{A}$对角化的过程:

    (1)解特征方程$|\lambda\boldsymbol{E}-\boldsymbol{A}|=0$得特征值$\lambda=\lambda_1(s_1\mbox{重}),\cdots,\lambda_m(s_m\mbox{重})$;
    
    (2)对每个特征值$\lambda_i,i=1,2,\cdots,m$,解齐次方程组$$(\lambda_i\boldsymbol{E}-\boldsymbol{A})\boldsymbol{x}=\boldsymbol{\theta}$$得到一个基础解系$\boldsymbol{\alpha}_{i1},\cdots,\boldsymbol{\alpha}_{ir_i}$.若有某个$i$使得$r_i<s_i$,则矩阵$\boldsymbol{A}$不可对角化;
    
    (3)当所有的$r_i=s_i,i=1,2,\cdots,m$,则令$$\boldsymbol{P}=\left(\begin{array}{cccccccccc}
            \boldsymbol{\alpha}_{11} & \cdots &\boldsymbol{\alpha}_{1s_1} & \boldsymbol{\alpha}_{21} & \cdots &\boldsymbol{\alpha}_{2s_2} & \cdots & \boldsymbol{\alpha}_{m1} & \cdots \boldsymbol{\alpha}_{ms_m} 
        \end{array}\right)$$即得
        $\boldsymbol{P}^{-1}\boldsymbol{A}\boldsymbol{P}=\boldsymbol{\varLambda} =\mathrm{diag}(\underbrace{\lambda_1,\cdots,\lambda_1}_{s_1\mbox{个}},\underbrace{\lambda_2,\cdots,\lambda_2}_{s_2\mbox{个}},\cdots,\underbrace{\lambda_m,\cdots,\lambda_m}_{s_m\mbox{个}})$.

    由于每个齐次线性方程组的基础解系都不是唯一的,因此$\boldsymbol{P}$的取法也不是唯一的,但是,由于$\boldsymbol{P}^{-1}\boldsymbol{A}\boldsymbol{P}$的主对角线上的元素是
    $\boldsymbol{A}$的全体特征值,因此除了主对角线上元素的次序不同外,$\boldsymbol{P}^{-1}\boldsymbol{A}\boldsymbol{P}$是唯一确定的。
\end{theorem}

\subsection{正交矩阵与施密特正交化方法}
\begin{definition}[向量内积]
    设$\boldsymbol{\alpha},\boldsymbol{\beta}$为$n$维向量,用列向量表示为$\boldsymbol{\alpha}={(a_1,a_2,\cdots,a_n)}^\mathrm{T},\boldsymbol{\beta}={(b_1,b_2,\cdots,b_n)}^\mathrm{T}$.
    若$\boldsymbol{\alpha},\boldsymbol{\beta}$为实向量,则称$a_1b_1+a_2b_2+\cdots+a_nb_n$为{\heiti 实内积};若$\boldsymbol{\alpha},\boldsymbol{\beta}$为复向量,则称$a_1b_1+a_2b_2+\cdots+a_nb_n$
    为$\boldsymbol{\alpha},\boldsymbol{\beta}$的{\heiti 复内积}.统称为{\heiti 向量的内积},记为$(\boldsymbol{\alpha},\boldsymbol{\beta})$,并称$\Vert\boldsymbol{\alpha}\Vert=\sqrt{(\boldsymbol{\alpha},\boldsymbol{\alpha})} $
    为向量$\boldsymbol{\alpha}$的{\heiti 长度}或{\heiti 模};称模为1的向量为{\heiti 单位向量}.
\end{definition}

\begin{definition}[向量夹角、向量正交]
    若$(\boldsymbol{\alpha},\boldsymbol{\beta})=0$,则称$\boldsymbol{\alpha}$和$\boldsymbol{\beta}${\heiti 正交}或{\heiti 垂直}.若$\boldsymbol{\alpha},\boldsymbol{\beta}$均为非零实向量,
    则称$\displaystyle{\arccos\left(\frac{(\boldsymbol{\alpha},\boldsymbol{\beta})}{\|\boldsymbol{\alpha}\|\cdot\|\boldsymbol{\beta}\|}\right)}$
    为向量$\boldsymbol{\alpha}$和$\boldsymbol{\beta}$的{\heiti 夹角}.
\end{definition}

\begin{definition}[正交向量组、法正交组]
    若一个不含零向量的向量组中的向量两两正交,则称该向量组为{\heiti 正交向量组};若一个正交向量组中的向量均为单位向量,则
    该向量组称为{\heiti 标准正交向量组},简称{\heiti 法正交组}。
\end{definition}

\begin{theorem}
    正交向量组必线性无关。
\end{theorem}

\begin{theorem}[施密特(Schmidt)正交化]
    由线性无关向量组$\boldsymbol{\alpha}_1,\boldsymbol{\alpha}_2,\cdots,\boldsymbol{\alpha}_n$可构造出与之等价的正交向量组
    $\boldsymbol{\xi }_1,\boldsymbol{\xi}_2,\cdots,\boldsymbol{\xi}_n$.并且$\boldsymbol{\xi}_i$可以表示成$\boldsymbol{\alpha}_1,\cdots,\boldsymbol{\alpha}_i,i=1,\cdots,n$
    的线性组合.
\end{theorem}

\begin{definition}[正交矩阵]
    若实方阵$\boldsymbol{A}$满足$\boldsymbol{A}^\mathrm{T}\boldsymbol{A}=\boldsymbol{E}$,则称$\boldsymbol{A}$为{\heiti 正交矩阵}。
\end{definition}

\begin{theorem}
    对于方阵$\boldsymbol{A}$,下列条件互为等价:
    \begin{enumerate}[(1)]
        \item $\boldsymbol{A}$为正交矩阵;
        \item $\boldsymbol{A}^\mathrm{T}=\boldsymbol{A}^{-1}$;
        \item $\boldsymbol{A}\boldsymbol{A}^\mathrm{T}=\boldsymbol{E}$;
        \item $\boldsymbol{A}$的列向量构成标准正交列向量组;
        \item $\boldsymbol{A}$的行向量构成标准正交行向量组.
    \end{enumerate}
\end{theorem}

\begin{theorem}
    设$\boldsymbol{A}$为$n$阶正交矩阵,$\lambda$为$\boldsymbol{A}$的特征值,$\boldsymbol{\alpha}$为$n$维列向量,则有
    \begin{enumerate}[(1)]
        \item ${|\boldsymbol{A}|}^2=1$;
        \item $(\boldsymbol{A}\boldsymbol{\alpha})^\mathrm{T}(\overline{\boldsymbol{A}\boldsymbol{\alpha}})=\boldsymbol{\alpha}^\mathrm{T}\overline{\boldsymbol{\alpha}}$;
        \item $|\lambda|=1$.
    \end{enumerate}
\end{theorem}

\subsection{实对称矩阵的对角化}
\begin{definition}[实对称矩阵]
    一个实矩阵$\boldsymbol{A}$具有对称性,即$\boldsymbol{A}^\mathrm{T}=\boldsymbol{A}$,称它为{\heiti 实对称矩阵}.
\end{definition}

\begin{theorem}
    实对称矩阵的特征值均为实数.
\end{theorem}

\begin{theorem}
    实对称矩阵的属于不同特征值的特征向量相互正交.
\end{theorem}

\begin{theorem}
    设有实$n$维单位列向量$\boldsymbol{\beta}$,则必能找到$n-1$个向量与$\boldsymbol{\beta}$一起构成由$n$个向量组成的标准正交向量组.
\end{theorem}

\begin{theorem}
    若$\boldsymbol{A}$是实对称矩阵,则存在同阶的正交矩阵$\boldsymbol{P}$使得$\boldsymbol{P}^\mathrm{T}\boldsymbol{A}\boldsymbol{P}$是实对角矩阵,
    从而实对称矩阵可对角化.
\end{theorem}